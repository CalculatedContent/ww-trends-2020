\documentclass[sigconf]{acmart}

%%
%% \BibTeX command to typeset BibTeX logo in the docs
\AtBeginDocument{%
  \providecommand\BibTeX{{%
    \normalfont B\kern-0.5em{\scshape i\kern-0.25em b}\kern-0.8em\TeX}}}

%% Rights management information.  This information is sent to you
%% when you complete the rights form.  These commands have SAMPLE
%% values in them; it is your responsibility as an author to replace
%% the commands and values with those provided to you when you
%% complete the rights form.
\setcopyright{acmcopyright}
\copyrightyear{2XXX}
\acmYear{2XXX}
\acmDOI{10.1145/1122445.1122XXX}

%% These commands are for a PROCEEDINGS abstract or paper.
\acmConference[Woodstock '18]{Woodstock '18: ACM Symposium on Neural
  Gaze Detection}{June 03--05, 2018}{Woodstock, NY}
\acmBooktitle{Woodstock '18: ACM Symposium on Neural Gaze Detection,
  June 03--05, 2018, Woodstock, NY}
\acmPrice{15.00}
\acmISBN{978-1-4503-XXXX-X/18/06}


%%
%% Submission ID.
%% Use this when submitting an article to a sponsored event. You'll
%% receive a unique submission ID from the organizers
%% of the event, and this ID should be used as the parameter to this command.
%%\acmSubmissionID{123-A56-BU3}

%%
%% The majority of ACM publications use numbered citations and
%% references.  The command \citestyle{authoryear} switches to the
%% "author year" style.
%%
%% If you are preparing content for an event
%% sponsored by ACM SIGGRAPH, you must use the "author year" style of
%% citations and references.
%% Uncommenting
%% the next command will enable that style.
%%\citestyle{acmauthoryear}

%%
%% end of the preamble, start of the body of the document source.
\begin{document}

%%
%% The "title" command has an optional parameter,
%% allowing the author to define a "short title" to be used in page headers.
\title{%
%XXX TITLE OF THIS EMPIRICAL PAPER
Predicting trends in generalization for state-of-the-art neural networks without access to training or testing data
}

%%
%% The "author" command and its associated commands are used to define
%% the authors and their affiliations.
%% Of note is the shared affiliation of the first two authors, and the
%% "authornote" and "authornotemark" commands
%% used to denote shared contribution to the research.

\author{Charles H. Martin}
\affiliation{%
  \institution{Calculation Consulting}
  \streetaddress{8 Locksley Ave, 6B}
  \city{San Francisco, CA 94122}
  \country{USA}}
\email{charles@CalculationConsulting.com}

\author{XXX SERENA}
\affiliation{%
  \institution{XXX}
  \streetaddress{XXX}
  \city{XXX}
  \country{XXX}}
\email{XXX}

\author{Michael W. Mahoney}
\affiliation{%
  \institution{ICSI and Department of Statistics, University of California at Berkeley}
  \streetaddress{XXX}
  \city{Berkeley, CA 94720}
  \country{USA}}
\email{mmahoney@stat.berkeley.edu}

%%
%% By default, the full list of authors will be used in the page
%% headers. Often, this list is too long, and will overlap
%% other information printed in the page headers. This command allows
%% the author to define a more concise list
%% of authors' names for this purpose.
%\renewcommand{\shortauthors}{Trovato and Tobin, et al.}

%%
%% The abstract is a short summary of the work to be presented in the
%% article.
\begin{abstract}
XXX.  ABSTRACT.
\end{abstract}

%%
%% The code below is generated by the tool at http://dl.acm.org/ccs.cfm.
%% Please copy and paste the code instead of the example below.
%%
\begin{CCSXML}
<ccs2012>
 <concept>
  <concept_id>10010520.10010553.10010562</concept_id>
  <concept_desc>Computer systems organization~Embedded systems</concept_desc>
  <concept_significance>500</concept_significance>
 </concept>
 <concept>
  <concept_id>10010520.10010575.10010755</concept_id>
  <concept_desc>Computer systems organization~Redundancy</concept_desc>
  <concept_significance>300</concept_significance>
 </concept>
 <concept>
  <concept_id>10010520.10010553.10010554</concept_id>
  <concept_desc>Computer systems organization~Robotics</concept_desc>
  <concept_significance>100</concept_significance>
 </concept>
 <concept>
  <concept_id>10003033.10003083.10003095</concept_id>
  <concept_desc>Networks~Network reliability</concept_desc>
  <concept_significance>100</concept_significance>
 </concept>
</ccs2012>
\end{CCSXML}

\ccsdesc[500]{Computer systems organization~Embedded systems}
\ccsdesc[300]{Computer systems organization~Redundancy}
\ccsdesc{Computer systems organization~Robotics}
\ccsdesc[100]{Networks~Network reliability}

%%
%% Keywords. The author(s) should pick words that accurately describe
%% the work being presented. Separate the keywords with commas.
\keywords{XXX, datasets, neural networks, gaze detection, text tagging}

%% A "teaser" image appears between the author and affiliation
%% information and the body of the document, and typically spans the
%% page.
%\begin{teaserfigure}
%  \includegraphics[width=\textwidth]{sampleteaser}
%  \caption{Seattle Mariners at Spring Training, 2010.}
%  \Description{Enjoying the baseball game from the third-base
%  seats. Ichiro Suzuki preparing to bat.}
%  \label{fig:teaser}
%\end{teaserfigure}

%%
%% This command processes the author and affiliation and title
%% information and builds the first part of the formatted document.
\maketitle


\section{Introduction}
\label{sxn:intro}

A common problem in machine learning (ML) 
%and artificial intelligence (AI) 
is to evaluate the quality of a given model.
A popular way to accomplish this
%, in particular in academic environments, 
is to train a model and then evaluate its training/testing error.
There are many problems with this approach.
The training/testing curves give very limited insight into the overall properties of the model; 
they do not take into account the (often large human and CPU/GPU) time for hyperparameter fiddling;
they typically do not correlate with other properties of interest such as robustness or fairness or interpretability; 
and so on.
A less well-known problem, but one that is increasingly important, in particular in industrial-scale artificial intelligence (AI), arises when the model \emph{user} is not the model \emph{developer}.
Here, one may not have access to either the training data or the testing data.
Instead, one may simply be given a model that has already been trained---a \emph{pretrained model}---and need to use it "as is,'' or to fine-tune and/or compress it and then use it.

Na\"{\i}vely---but in our experience commonly, among ML practitioners and ML theorists---if one does not have access to training or testing data, then one can say absolutely nothing about the quality of a ML model.
This may be true in worst-case theory, but models are used in practice, and there is a need for a \emph{practical theory} to guide that practice.
Moreover, if ML is to become an industrial process, then that process will become siloed: some groups will gather data, other groups will develop models, and other groups will use those models.
Users of models can not be expected to know the precise details of how models were built, the specifics of data that were used to train the model, what was the loss function or hyperparameter values, how precisely the model was regularized,~etc.

Moreover, for many large scale, practical applications, there is no obvious way to define an ideal test metric. 
For example, models that generate fake text or conversational chatbots may use a proxy, like perplexity, as a test metric.
In the end, however, they really require human evaluation. 
Alternatively, models that cluster user profiles, which are widely used in areas such as marketing and advertising, are unsupervised and have no obvious labels for comparison and/or evaluation.
In these and other areas, ML objectives can be poor proxies for downstream goals.

Most importantly, in industry, one faces unique practical problems such as: do we have enough data for this model? 
Indeed, high quality, labeled data can be very expensive to acquire, and this cost can make or break a project.
Methods that are developed and evaluated on any well-defined publicly-available coprus of data, no matter how large or diverse or interesting, are clearly not going to be well-suited to address problems such as this.
It is of great practical interest to have metrics to evaluate the quality of a trained model---in the absence of training/testing data and without any detailed knowledge of the training/testing process.  
We seek a practical theory for pretrained models which can predict how, when, and why such models can be expected to perform well or~poorly.

In this paper, we present and evaluate quality metrics for pretrained deep neural network (DNN) models, and we do so at scale.
We consider a large suite of hundreds of publicly-available models, mostly from computer vision (CV) and natural language processing (NLP).
%
By now, there are many such state-of-the-art models that are publicly-available, e.g., 
there are now hundreds of pretrained models in CV ($\ge 500$) and NLP ($\approx 100$).%
\footnote{When we began this work in 2018, there were fewer than tens of such models; now in 2020, there are hundreds of such models; and we expect that in a year or two there will be an order of magnitude or more of such models.}
These provide a large corpus of models that by some community standard are state-of-the-art.%
\footnote{Clearly, there is a selection bias or survivorship bias here---people tend not to make publicly-available their poorly-performing models---but these models are things in the world that (like social networks or the internet) can be analyzed for their properties.}
Importantly, all of these models have been trained by someone else and have been viewed to be of sufficient interest/quality to be made publicly-available; and, for all of these models, we have no access to training data or testing data, and we have no knowledge of the training/testing protocols. 

The \emph{quality metrics} we consider are based on the spectral properties of the layer weight matrics.
They are based on norms of weight matrices (such norms have been used in traditional statistical learning theory to bound capacity and construct regularizers) and/or parameters of power law (PL) fits of the eigenvalues of weight matrices (such PL fits are based on statistical mechanics approaches to DNNs).
Note that, while we use traditional norm-based and PL-based metrics, our goals are not the traditional goals.
Unlike more common ML approaches, \emph{we do not seek a bound on the generalization} (e.g., by evaluating training/test error during training), \emph{we do not seek a new regularizer}, and \emph{we do not aim to evaluate a single model} (e.g., as with hyperparmeter optimization).% 
\footnote{One could of course use these techniques to improve training, and we have been asked about that, but we are not interested in that here. Our main goal here is to use these techniques to evaluate properties of state-of-the-art pretrained DNN models.}
Instead, we want to examine different models across common architecture series, and we want to compare models between different architectures themselves, and in both cases, we ask:
\begin{quote}
\emph{Can we predict trends in the quality of pretrained DNN models without access to training or testing data?}  
\end{quote}


%\begin{itemize}[leftmargin=*]
%\item 
%First, 
%Motivated by practical problems, we formulate the question ``Can one predict trends in the quality of state-of-the-art neural networks without access to training or testing data''
%\item
%Second, 
%To answer this question, we analyze hundreds of publicly-available pretrained models, including older and current state-of-the-art models in CV and NLP.
%\end{itemize}

To answer this question, we analyze hundreds of publicly-available pretrained state-of-the-art CV and NLP models. 
Here is a summary of our main results.
\begin{itemize}
\item
Norm-based metrics do a reasonably good job at predicting quality trends in well-trained CV/NLP models.
%, i.e., they can be used to discriminate between ``good-better-best'' models.
\item
Norm-based meterics, however, may give spurious results, such as \emph{Scale\  Collapse} for poorly-trained modeles (i.e. models trained without enough data, etc.).  
\item 
PL-based metrics do much better--quanttiatviely--at predicting quality trends in pretrained models. That is, they are  quantitatively better at discriminating among a series of ``good-better-best", and are qualitatively better at discriminating good-versus-bad models.
\item 
PL-based metrics can be used to characterize fine-scale model properties (including layer-wise \emph{Correlation Flow}) in both well-trained and /or poorly-trained models), and can be used to evaluate model enhancements (i.e. distillation, finetuning, etc.)
\end{itemize}

\noindent
We emphasize that our goal is a practical theory to predict trends in the quality of state-of-the-art DNN models, i.e., not to make a statement about every publicly-available model.
We have examined hundreds of models, and we identify general trends, but we also highlight interesting exceptions.
%%Several of the most interesting are described below.

\paragraph{The WeightWatcher Tool.}

All of our computations were performed with the publicly-available \emph{WeightWatcher} tool (version 0.2.7)~\cite{weightwatcher_package}.
To be fully reproducible, we only examine publically-available, pretrained models, and we also provide all Jupyter and Google Colab notebooks used in an accompanying github repository.%
\footnote{\url{https://github.com/CalculatedContent/kdd2020} \michael{TO BE ANONYMIZED.}}
See Appendix~\ref{sxn:appendix} for details on how to reproduce all results.


\paragraph{Organization of this paper.}

We start in Section~\ref{sxn:background} and Section~\ref{sxn:methods} with background and an overview of our general approach.
In Section \ref{sxn:cv}, we study three well-known widely-available DNN CV architectures (the VGG, ResNet, and DenseNet series of models); and we provide an illustration of our basic methodology, both to evaluate the different metrics against reported test accuracies and to use quality metrics to understand model properties.
Then, in Section~\ref{sxn:nlp}, we look at several variations of a popular NLP DNN architecture (the OpenAI GPT and GPT2 models); and we show how model quality and properties vary between several variants of GPT and GPT2, including how metrics behave similarly and differently.
Then, in Section \ref{sxn:all_cv_models}, we present results based on an analysis of hundreds of pretraind DNN models, showing how well each metric predicts the reported test accuracies, and how the PL-based metrics perform remarkably well.
Finally, in Section~\ref{sxn:conc}, we provide a brief discussion and conclusion.




\section{Background and Related Work}
\label{sxn:background}


To our knowledge, there is very little work on the particular question we are addressing: namely, how to predict, in a theoretically-principled manner, the quality of large-scale state-of-the-art NNs, and to do so without access to training data or testing data or details of the training protocol, etc.
Our work is, however, loosely related to several other lines of work, and we briefly discuss them here.

\paragraph{Statistical mechanical theory of NNs.}

XXX.
Cite our stuff:
\cite{MM17_TR},
\cite{MM18_TR},
\cite{MM19_HTSR_ICML},
\cite{weightwatcher_package}
\cite{MM19_KDD},
\cite{MM20_SDM},
\cite{MM20_unpub_work}.C
Cite also Ganguli review and maybe other stuff.
\charles{Gangulis work is not really close to what we are doing, although I get the politics.   The closest useful work would be the older work by Bouchaud and Potters, which discusses the statisical mechanics of heavy tailed and strongly correlated systems form the 90s}

XXX.
Distinguish between what we will call a
\emph{phenomenological theory}
(that describes empirical relationship of phenomena to each other, in a way which is consistent with fundamental theory, but is not directly derived from that theory)
and what can be called a 
\emph{first principles theory} 
(that is applicable to tiny things but has no hope of scaling up).
\charles{it is the oppposite...ab initio theory scales quite well...it is the spherical cow models of physics and ML that do not scale. What we have is a semi-empirical theory.  We use real theory, but require empirical input, at least in the new stat mech work.  What we have introduced is a phenomenology, which to me is different from a semi-empirical or phenomenological theory  }
\footnote{In most areas where there are complex highly-engineered systems (beyond complex AI/ML systems), one used phenomenological theory rather than first principles theory.  For example, one does not try to solve the Schr\"odinger equation if one is interested in building a bridge or an airplane.}

\paragraph{Norm-based capacity control theory.}
XXX.  MOST IRRELEVANT, BUT LIAO AND OUR SDM ARE RELATED.  
XXX.  MAYBE BEAT ON INFINITELY WIDE OR SOMETHING ELSE.

\paragraph{Practical problems poorly addressed by theory.}
There are many very practical problems in ML that are poorly addressed by existing theory and that either motivated our work or should be addressable by our techniques.
Here are several.
\begin{itemize}
\item
\textbf{Simplicity, or lack of,  accuracy metrics.}
\item
\textbf{Information leakage in the production pipeline}
\item
\textbf{Cost of acquring labeled data.}
\end{itemize}
Importantly, there are many examples in ML where (as a practical matter) there is no reliable notion of accuracy, e.g., when generating fake text, when developing self driving cars, or creating realistic chatbots.\footnote{i.e. current chatbots use perplexity as a proxy for passing a Turing test}
%, and when distilling a reliable model in some way to obtain comparable training/test quality but that damages the model in some other subtle way. 
%\michael{That last example is awkward.  It would be good to have a better example and plant seeds for model distillation elsewhere.}

%KDD% \vspace{-1mm}
\section{Conclusion}
\label{sxn:conc}
%\vspace{-1mm}

We have developed (based on strong theory) and evaluated (on a large corpus of publicly-available pretrained models from CV and NLP) methods to predict trends in the quality of state-of-the-art neural networks---without access to training or testing data.
Prior to our work, it was not obvious that norm-based metrics would perform well to predict trends in quality \emph{across} models (as they are usually used \emph{within} a given model or parameterized model class, e.g., to bound generalization error or to construct regularizers).
Our results are the first to demonstrate that they can be used for this important practical problem.
That PL-based metrics perform better (than norm-based metrics) should not be surprising---at least to those familiar with the statisical mechanics of heavy tailed and strongly correlated systems~\cite{BouchaudPotters03, SornetteBook, BP11, bun2017} (since our use of PL exponents is designed to capture the idea that well-trained models capture correlations over many size scales in the data).
Again, though, our results are the first to demonstrate this.
It is also gratifying that this approach can be used to provide fine-scale insight (such as rationalizing the flow of correlations or the collapse of size scale) throughout a network. 

We conclude with a few comments on what a \emph{practical theory} of DNNs should look like.
To do so, we distinguish between two types of theories:
\emph{non-empirical or analogical theories}, in which one creates, often from general principles, a very simple toy model that can be analyzed rigorously, and one then argues that the model is relevant to the system of interest; and 
\emph{semi-empirical theories}, in which there exists a rigorous asymptotic theory, which comes with parameters, for the system of interest, and one then adjusts or fits those parameters to the finite non-asymptotic data.
A drawback of the former approach is that it typically makes very strong assumptions on the data, and the strength of those assumptions can limit the practical applicability of the theory.
Much of the work on the theory of DNNs focuses on the former type of theory.
Our approach focuses on the latter type of theory.
Our results, which are based on our \emph{use} of sophisticated statistical mechanics theory to solve important practical DNN problems, suggests that our approach should be of interest more generally for those interested in developing a practical DNN theory.




%%
%% The acknowledgments section is defined using the "acks" environment
%% (and NOT an unnumbered section). This ensures the proper
%% identification of the section in the article metadata, and the
%% consistent spelling of the heading.
\begin{acks}
MWM would like to acknowledge ARO, DARPA, IARPA, NSF, and ONR for providing partial support of this work.
\end{acks}

%%
%% The next two lines define the bibliography style to be used, and
%% the bibliography file.
\bibliographystyle{ACM-Reference-Format}
%\bibliography{gen_gap}
%\bibliography{dnns}
\bibliography{dnns,gen_gap}

\end{document}

