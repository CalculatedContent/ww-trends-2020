\section{Conclusion}
\label{sxn:conc}


XXX.  PUT CONCLUSION HERE AND WEAVE IN COMMENTS FROM BELOW.


Some other comments that we need to weave into a narrative eventually after later sections are written:
\begin{itemize}
\item
GPT versus GPT2.
What happens when we don't have enough data?
This is the main question, and we can use out metrics to evaluate that, but we also get very different results for GPT versus GPT2.
\item
The spectral norm is a regularizer, used to distinguish good-better-best, not a quality metric.
For example, it can ``collapse,'' and for bad models we can have small spectral norm.
So, it isn't really a quality metric.
\item
One question that isn't obvious is whether regularization metrics can be used as quality metrics.
One might think so, but the answer isn't obviously yes.
We show that the answer is No.
A regularizer is designed to select a unique solution from a non-unique good-better-best.
Quality metrics can also distinguish good versus bad.
\item
(We should at least mention this is like the statistical thing where we evaluate which model is better, as oposed to asking if a given model is good, I forget the name of that.)
\item
There are cases where the model is bad but regularization metric doesn't tell you that.
Quality should be correlated in an empirical way.
Correlated with good-better-best; but also tell good-bad.
\item
Question: why not use regularier for quality?
Answer: A regularizer selects from a given set of degenerate models one which is nice or unique.
It doesn't tell good versus bad, i.e., whether that model class is any good.
\item
Thus, it isn't obvious that norm-based metrics should do well, and they don't in general.
\item
We give examples of all of these: bad data; defective data; and distill models in a bad way.
(Of course, bad data means bad model, at least indirectly, since the quality of the data affects the properties of the model.)
\item
We can select a model and change it, i.e., we don't just do hyperparameter fiddling.
\end{itemize}

%Perhaps our most improtant contribution, however, is just doing things different, etc. ...
