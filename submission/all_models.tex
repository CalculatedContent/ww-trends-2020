\section{Comparing Hundreds of CV Models}
\label{sxn:all_cv_models}

In this section, we summarize results from a large-scale analysis of hundreds of CV models, including models developed for image classification, segmentation, and a range of related tasks.  
Our aim is to complement the detailed results from Sections~\ref{sxn:cv} and~\ref{sxn:nlp} by providing broader conclusions. % about norm-based and PL-based metrics.
The models we consider have been pretrained on nine datasets, including ImageNet-1K, and CIFAR-10, CIFAR-100, Street View House Numbers (SVHN), Caltech-UCSD Birds-200-2011 (CUB-200-2011), Pascal VOC2012, ADE20K, Cityscapes, and Common Objects in Context (COCO). 
%%The pretrained models and their accuracy metrics are summarized in the osmr github~\cite{charles-add-link}.
We provide more details about our experimental setup in Appendix~\ref{sxn:appendix}.

We choose simple linear regression to analyze the relationship between quality metrics (computed with the \emph{WeightWatcher} tool) and traditional accuracy metrics (publicly-reported from a test set).
% (we avoid polynomial regressions as they are more prone to overfitting and does not make economic sense).
We regress the metrics on the Top1 (and Top5) reported errors (as dependent variables).  
These include Top5 errors for the ImageNet-1K model, percent error for the CIFAR-10/100, SVHN, CUB-200-2011 models, and Pixel accuracy (Pix.Acc.) and Intersection-Over-Union (IOU) for other models.
We regress them individually on each of the norm-based and PL-based metrics, as described above.

Our results are summarized in Table~\ref{table:results}.
For the mean, larger $R^{2}$ and smaller $MSE$ are desirable; and for the standard deviation, smaller values are desirable.
Taken as a whole, over the entire corpus of data, PL-based metrics are somewhat better for both the $R^{2}$ mean and standard deviation; and PL-based metrics are much better for $MSE$ mean and standard deviation.
These (and other) results suggest our conclusions from Sections~\ref{sxn:cv} and~\ref{sxn:nlp} hold much more generally, and they suggest ovbious questions for future work.


\begin{table}[t]
\small
\begin{center}
%\begin{tabular}{|p{1in}|c|c|c|c|}
\begin{tabular}{|p{0.75in}|c|c|c|c|}
\hline
%    & Frobenius Norm & Spectral Norm & Weighted Alpha & Alpha-Norm \\
%    & $\langle\log\Vert\mathbf{W}\Vert_{F}\rangle$ & $\langle\log\Vert\mathbf{W}\Vert_{\infty}\rangle$ & $\langle\hat{\alpha}=\alpha\log\lambda_{max}\rangle$ & $\langle\log\Vert\mathbf{X}\Vert^{\alpha}_{\alpha}\rangle$ \\
Series        & $\log\Vert\cdot\Vert_{F}$ & $\log\Vert\cdot\Vert_{\infty}$ & $\hat{\alpha}$ & $\log\Vert\cdot\Vert^{\alpha}_{\alpha}$ \\
\hline
$R^{2}$ (mean) & 0.63 &  0.55 & \textbf{0.64} & \textbf{0.64} \\
$R^{2}$ (std)  & 0.34 &  0.36 & \textbf{0.29} &          0.30 \\
\hline
$MSE$ (mean)   & 4.54 &  9.62 &          3.14 & \textbf{2.92} \\
$MSE$ (std)    & 8.69 & 23.06 &          5.14 & \textbf{5.00} \\
\hline
\end{tabular}
\end{center}
\caption{Comparison of linear regression fits for different average Log Norm and Weighted Alpha metrics across 5 CV datasets, 17 architectures, covering 168 (out of 309) different pretrained DNNs.  
         We include regressions only for architectures with 4 or more data points, and which are postively correlated with test error.
         These results can be readily reproduced using the Google Colab notebooks (see Appendix~\ref{sxn:appendix}). 
        }
\label{table:results}
\end{table}



