%KDD% \vspace{-1mm}
\section{Background and Related Work}
\label{sxn:background}
%\vspace{-1mm}

Most theory for DNNs is applied to small toy models and assumes access to data.
There is very little work asking how to predict, in a theoretically-principled manner, the quality of large-scale state-of-the-art DNNs, and how to do so without access to training data or testing data or details of the training protocol, etc.
%Our 
%approach  %%  work 
%is, however, 
%%% loosely 
%related to 
%two  %%  several 
%other lines of work.
Similarly, there is very little work on performing weight matrix analysis.
Here are several other lines of work most related to our approach.


%KDD% \vspace{-1mm}
\paragraph{Statistical mechanics theory for DNNs.}

Statistical mechanics ideas have long had influence on DNN theory and practice~\cite{EB01_BOOK, MM17_TR, BKPx20}; and 
our best-performing metrics (those using fitted PL exponents) are based on statistical mechanics~\cite{MM17_TR, MM18_TR, MM19_HTSR_ICML, MM19_KDD, MM20_SDM}, in particular the recently-developed \emph{Theory of Heavy Tailed Self Regularization (HT-SR)}~\cite{MM18_TR, MM19_HTSR_ICML, MM20_SDM}.  
See Appendix~\ref{sxn:theory-review-appendix} for a brief overview of this theory.
We emphasize that the way in which we (and HT-SR Theory) \emph{use} statistical mechanics theory is quite different than the way it is more commonly formulated.
Several very good overviews of the more common approach are available~\cite{EB01_BOOK, BKPx20}.
We \emph{use} statistical mechanics in a broader sense, drawing upon techniques from quantitative finance and random matrix theory.  % (RMT).
Thus, much more relevant for our methodological approach is older work of Bouchaud, Potters, Sornette, and coworkers~\cite{BouchaudPotters03, SornetteBook, BP11, bun2017} on the statistical mechanics of heavy tailed and strongly correlated systems.
% Probably best not to cite at this point %\charles{, and current work (in progress)\cite{blog}}


%KDD% \vspace{-1mm}
\paragraph{Norm-based capacity control theory.}

There is also a large body of work on using norm-based metrics to bound generalization error~\cite{NTS15, BFT17_TR, LMBx18_TR}.
In this area, theoretical work aims to prove generalization bounds, and applied work uses these norms to construct regularizers to improve training.
While we do find that norms provide relatively good quality metrics, at least for distinguishing among (moderately) well-trained versus 
very-well-trained models, of increasing depth and accuracy, 
we are not interested in proving generalization bounds or developing new regularizers.

\paragraph{Weight matrix analysis.}
To our knowledge, our work, starting in 2018 with the \emph{WeightWatcher} tool~\cite{weightwatcher_package}, is the first to perform a detailed analysis of the weight matrices of DNN models~\cite{MM18_TR, MM19_HTSR_ICML, MM20_SDM}.
Subsequent to the release of the initial version of this paper, we became aware of two other works that were posted to the arXiv in 2020 within weeks of the initial arXiv version of this paper~\cite{EJRUY20_TR,UKGBT20_TR}.
Both of these papers validate our basic result that one can gain substantial insight into model quality by examining weight matrices without access to any training or testing data.
However, both of these papers consider smaller models drawn from a much narrower range of applications than we consider, and previous results in HT-SR Theory suggest that many of the insights from these smaller models may not extend to the state-of-the-art CV and NLP models we consider.



%% \paragraph{Practical problems need a practical theory.}
%% \nred{Exiting ML theory does not address many practical AI problems that our techniques directly target.}
%% %There are many very practical problems in ML that are poorly addressed by existing theory and that either motivated our work or should be addressable by our techniques.
%% Here are several.
%% \begin{itemize}[leftmargin=*]
%% \item
%% \textbf{Simplicity, or lack of,  accuracy metrics.}
%% \michael{XXX.  Put in two or at most three sentences.}
%% \item
%% \textbf{Information leakage in the production pipeline.}
%% \michael{XXX.  Put in two or at most three sentences.}
%% \item
%% \textbf{Cost of acquiring labeled data.}
%% \michael{XXX.  Put in two or at most three sentences.}
%% \end{itemize}
%% 
%% \charles{see comments in tex file}\michael{Let's touch base, not sure what to do.}
%% % we address this above
%% %Importantly, there are many examples in ML where (as a practical matter) there is no reliable notion of accuracy, e.g., when generating fake text or realistic chatbots, developing self driving cars, or just clustering user profiles.
%% %\footnote{For example, current chatbots use perplexity as a proxy for passing a Turing test.}
%% %, and when distilling a reliable model in some way to obtain comparable training/test quality but that damages the model in some other subtle way. 
%% %\michael{That last example is awkward.  It would be good to have a better example and plant seeds for model distillation elsewhere.}
