
In many practical applications, one works with deep neural networks (DNNs) models trained by someone else.
For such \emph{pretrained models}, one typically does not have access to training data or test data, 
and, moreover,  does not know many detals aout the  model, such as
the specifics of the training data, the loss function, the hyperparameter values, etc.
Given just one or more pretrained models, can one say anything about the expected performance or quality of the models ?
Here, we present and evaluate empirical quality metrics for pretrained neural network models at scale.
%%Our metrics are drawn from traditional statistical learning theory (e.g., norm-based capacity control metrics) as well as Heavy-Tailed Random Matrix Theory (HT-RMT), as used in the recently-developed Theory of Heavy-Tailed Self Regularization (e.g., fitted power law metrics used to characterize the degree of strong correlations in trained models).
%%The most promising are metrics drawn from traditional statistical learning theory (e.g., norm-based capacity control metrics) as well as metrics (e.g., fitted power law metrics used to characterize the degree of strong correlations in a system) derived from the recently-developed Theory of Heavy-Tailed Self Regularization (HT-SR).
%The most promising are norm-based metrics
% (such as those used to provide capacity control in traditional statistical learning theory) 
%and metrics based on fitting eigenvalue distributions to truncated power law (PL) distributions 
%(which derive from statistical mechanics, in particular the recently-developed Theory of Heavy-Tailed Self Regularization, and which characterize the strength of correlations in a DNN). 
%
Using the publicly-available \emph{WeightWatcher} tool, we analyze hundreds of publicly-available pretrained models, including older and current state-of-the-art models in computer vision (CV) and natural language processing (NLP).
We examine both familiar norm-based capacity control metrics (the Frobenius and  Spectral norms). as well as
new Power-Law (PL) based metrics,  the Weighted Alpha ($\alpha$) and the $\alpha$-(Shatten) Norm,
from the recently-developed Theory of Heavy-Tailed Self Regularization (HT-SR).
We find that norm-based metrics do a reasonably good job at predicting quality trends in well-trained models, i.e., they can discriminate amoing ``good-better-best''  models.
On the other hand, for poorly trained models, i.e., to distinguish ``good-versus-bad'' models---which, arguably is the point of needing quality metrics---norm-based metrics can qualitatively fail.  
We also find that PL-based metrics do much better---quantitatively better at discriminating aming ``good-better-best'' models, and qualitatively better at discriminating ``good-versus-bad'' models.
PL-based metrics can also be used to characterize fine-scale properties of models, e.g., understanding layer-wise \emph{correlation flow}, and evaluate post-training modifications such as model distillation, increasing the data set size, and other model improvements.

