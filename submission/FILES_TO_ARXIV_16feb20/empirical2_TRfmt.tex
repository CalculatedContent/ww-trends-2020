\documentclass[11pt]{article}

\addtolength{\textwidth}{1.4in}
\addtolength{\oddsidemargin}{-0.5in}
\addtolength{\evensidemargin}{-0.5in}
\addtolength{\topmargin}{-1.0in}
\addtolength{\textheight}{1.7in}
\newlength{\defbaselineskip}
\setlength{\defbaselineskip}{\baselineskip}

\usepackage{algorithm}
\usepackage{algpseudocode}
\usepackage{framed}
\usepackage{amssymb}
\usepackage{amsfonts}
\usepackage{amsmath}
\usepackage{graphicx}
\usepackage{url}
\usepackage{rotating}
\usepackage{multirow}
\usepackage{color}
\usepackage{xcolor}
\usepackage{paralist}

\usepackage{subfigure}

\usepackage{longtable} 
\usepackage{makecell}

\usepackage[multiple]{footmisc}
\usepackage[section]{placeins}

\usepackage{hyperref}
\hypersetup{
     colorlinks   = true,
     linkcolor    = blue,
     citecolor    = green
}


\newcommand{\fix}[1]{\textcolor{red}{#1}}
\newcommand{\comment}[1]{\textcolor{blue}{#1}}
\newcommand{\awk}[1]{\textcolor{green}{#1}}

\newcommand{\argmin}{\text{argmin}}
\newcommand{\Probab}[1]{\mbox{}{\bf{Pr}}\left[#1\right]}
\newcommand{\Expect}[1]{\mbox{}{\bf{E}}\left[#1\right]}
\newcommand{\ExpectBracket}[1]{\mbox{}\langle#1\rangle}

\newcommand {\nred}[1]{{\color{red}\sf{[#1]}}}
\newcommand {\ngreen}[1]{{\color{green}\sf{[#1]}}}
\newcommand {\ncyan}[1]{{\color{cyan}\sf{[#1]}}}
\newcommand {\michael}[1]{{\color{red}\sf{[michael: #1]}}}
\newcommand {\charles}[1]{{\color{blue}\sf{[charles: #1]}}}
\newcommand {\serena}[1]{{\color{orange}\sf{[charles: #1]}}}

\usepackage[normalem]{ulem}




\begin{document}

\title{%
Predicting trends in the quality of state-of-the-art neural networks without access to training or testing data
}

\author{%
Charles H. Martin\thanks{Calculation Consulting, 8 Locksley Ave, 6B, San Francisco, CA 94122, \texttt{charles@CalculationConsulting.com}.} 
\and 
Tongsu (Serena) Peng\thanks{Calculation Consulting, 8 Locksley Ave, 6B, San Francisco, CA 94122, \texttt{serenapeng7@gmail.com}.}
\and
Michael W. Mahoney\thanks{ICSI and Department of Statistics, University of California at Berkeley, Berkeley, CA 94720, \texttt{mmahoney@stat.berkeley.edu}.}
}

\date{}
\maketitle



\begin{abstract}

In many practical applications, one works with deep neural networks (DNNs) models trained by someone else.
For such \emph{pretrained models}, one typically does not have access to training data or test data, 
and, moreover,  does not know many detals aout the  model, such as
the specifics of the training data, the loss function, the hyperparameter values, etc.
Given just one or more pretrained models, can one say anything about the expected performance or quality of the models ?
Here, we present and evaluate empirical quality metrics for pretrained neural network models at scale.
%%Our metrics are drawn from traditional statistical learning theory (e.g., norm-based capacity control metrics) as well as Heavy-Tailed Random Matrix Theory (HT-RMT), as used in the recently-developed Theory of Heavy-Tailed Self Regularization (e.g., fitted power law metrics used to characterize the degree of strong correlations in trained models).
%%The most promising are metrics drawn from traditional statistical learning theory (e.g., norm-based capacity control metrics) as well as metrics (e.g., fitted power law metrics used to characterize the degree of strong correlations in a system) derived from the recently-developed Theory of Heavy-Tailed Self Regularization (HT-SR).
%The most promising are norm-based metrics
% (such as those used to provide capacity control in traditional statistical learning theory) 
%and metrics based on fitting eigenvalue distributions to truncated power law (PL) distributions 
%(which derive from statistical mechanics, in particular the recently-developed Theory of Heavy-Tailed Self Regularization, and which characterize the strength of correlations in a DNN). 
%
Using the publicly-available \emph{WeightWatcher} tool, we analyze hundreds of publicly-available pretrained models, including older and current state-of-the-art models in computer vision (CV) and natural language processing (NLP).
We examine both familiar norm-based capacity control metrics (the Frobenius and  Spectral norms). as well as
new Power-Law (PL) based metrics, the PL exponent $\alpha$ and the Weighted Alpha ($\hat{\alpha}$) metrics,
from the recently-developed Theory of Heavy-Tailed Self Regularization (HT-SR).
We also introduce the $\alpha$ (Shatten) Norm metric.
We find that norm-based metrics correlate well with reported test accuracies of well-trained models
across nearly all CV architecture series.
PL-based metrics can also be used to characterize fine-scale properties of these models, and
we define the layer-wise \emph{Correlation Flow} as new quality assesment.
We also find that PL-based metrics do much better--quantitatively-- at discriminating among
a series of ``good-better-best'' models, and qualitatively better at discriminating ``good-versus-bad'' models.
On the other hand, we find that norm-based metrics can not distinguish ``good-versus-bad'' models--which, 
arguably is the point of needing quality metrics.  Indeed, they may give spurious results.
We show how poorly-trained (and/or poorly-finetuned) models may exhbit both \emph{Scale Collapse}
and unsually large PL exponents $\alpha\gg 6$.  In particular, we show that the \emph{WeightWatcher} tool
can be used to identify when a pretrained DNN has problems that can not be detected simply
by examining the test accuracy.





\end{abstract}



\section{Introduction}
\label{sxn:intro}

A common problem in machine learning (ML) 
%and artificial intelligence (AI) 
is to evaluate the quality of a given model.
A popular way to accomplish this
%, in particular in academic environments, 
is to train a model and then evaluate its training and/or testing error.
There are many problems with this approach.
Well-known problems with just examining training/testing curves include that 
they give very limited insight into the overall properties of the model, 
they do not take into account the (often extremely large human and CPU/GPU) time for hyperparameter fiddling,
they typically do not correlate with other properties of interest such as robustness or fairness or interpretability, 
%XXX SOMETHING ELSE, 
and so on.
A somewhat less well-known problem, but one that is increasingly important (in particular in industrial-scale ML---where the \emph{users} of models are not the \emph{developers} of the models) is that one may access to neither the training data nor the testing data.
Instead, one may simply be given a model that has already been trained---we will call such an already-trained model a \emph{pre-trained model}---and be told to use it.

Na\"{\i}vely---but in our experience commonly, among both ML practitioners and ML theorists---if one does not have access to training or testing data, then one can say absolutely nothing about the quality of a ML model.
This may be true in worst-case theory, but ML models are used in practice, and there is a need for theory to guide that practice.
Moreover, if ML is to become an industrial process, then the process will become siloed: some groups will gather data, other groups will develop models, and still other groups will use those models.
The users of models can not be expected to know the precise details of how the models were built, the specifics of the data that were used to train the model, what the loss of values of the hyperparameters were, how precisely the model was regularized, etc.
%
Having metrics to evaluate the quality of a ML model in the absence of training and testing data and without any detailed knowledge of the training and testing process---indeed, having theory for pre-trained models, to predict how, when, and why such models can be expected to perform well or poorly---is clearly of interest.

In this paper, we present and apply techniques to evaluate the generalization properties of large-scale state-of-the-art pre-trained neural network (NN) models.%
\footnote{We reiterate: One could use these techniques to improve training, and we have been asked about that, but we are not interested in that here. Our main goal here is to use these techniques to evaluate properties of state-of-the-art pre-trained NN models.}
To do so, we consider a large suite of publicly-available models from computer vision (CV) and natural language processing (NLP).
%
By now, there are many such state-of-the-art models that are publicly-available, e.g., 
there are now hundreds of pre-trained models in CV ($\ge 500$) and NLP ($\approx 100$).%
\footnote{When we began this work in 2018, there were fewer than tens of such models; now in 2020, there are hundreds of such models; and we expect that in a year or two there will be an order of magnitude or more of such models.}
These provide a large corpus of models that by some community standard are state-of-the-art.%
\footnote{Clearly, there is a selection bias or survivorship bias here---people tend not to make publicly-available their poorly-performing models---but these models are things in the world that (like social networks or the internet) can be analyzed for their properties.}
XXX.  MORE DETAILS.
Importantly, for all of these models, we have no access to training data or testing data.

XXX.  LIST PLACES WHERE THEY ARE AVAIALBLE, HERE OR IN NEXT SECTION.

In more detail, our main contributions are the following.
\begin{itemize}
\item XXX TECHNICAL THING 1
\item XXX TECHNICAL THING 2
\item XXX TECHNICAL THING 3
\end{itemize}
Perhaps our most improtant contribution, however, is just doing things different, etc. ...


\section{Background and Related Work}
\label{sxn:background}




Here we will cite and discuss related work, including


\cite{MM17_TR},
\cite{MM18_TR},
\cite{MM19_HTSR_ICML},
\cite{weightwatcher_package}
\cite{MM19_KDD},
\cite{MM20_SDM},
\cite{MM20_unpub_work},


\section{Methods}
\label{sxn:methods}


Let us write the Energy Landscape (or optimization function, parameterized by $\mathbf{W}_{l}$s and $\mathbf{b}_{l}$s) for a typical DNN with $L$ layers, with activation functions $h_{l}(\cdot)$, and with $N\times M$ weight matrices $\mathbf{W}_{l}$ and biases $\mathbf{b}_{l}$, as:
\begin{equation}
E_{DNN}=h_{L}(\mathbf{W}_{L}\times h_{L-1}(\mathbf{W}_{L-1}\times h_{L-2}(\cdots)+\mathbf{b}_{L-1})+\mathbf{b}_{L})  .
\label{eqn:dnn_energy}
\end{equation}
We assume we are given several pretrained Deep Neural Networks (DNNs), e.g., as part of an architecture series.
The models have been trained on (unspecified and unavailable) labeled data $\{d_{i},y_{i}\}\in\mathcal{D}$, using Backprop, by minimizing some (also unspecified and unavailable) loss function $\mathcal{L}()$. 
We expect that most well-trained, production-quality models will employ one or more forms of on regularization, such as Batch Normalization, Dropout, etc., and many will also contain additional structure such as Skip Connections, etc. 
Here, we will ignore these details, and will focus only on the weight matrices. 


\paragraph{DNN Quality Metrics.}

Each DNN layer contains one or more layer 2D weight matrices $\mathbf{W}_{L}$, and/or 2D feature maps $\mathbf{W}_{i,L}$ extracted from 2D Convolutional layers. 
(For notational convenience, we may drop the $i$ and/or $i,l$ subscripts below.) 
%
\charles{hard to confer on this.  I tried to state this as simply as possible.  We have never stated this before, and I think its going to be confusing.  We at least need to explain why we are looking at the average log norm. Otherwise we are forcing the reader to look back, and the paper is hard to read as a standalone entity. }
\michael{
XXX.  PROBABLY WE DO NOT WANT THE FOLLOWING.  INSTEAD, JUST PRESENT METRICS AND POINT TO PREVIOUS PAPERS FOR JUSTIFICATION.  E.G., WE DONT WANT TO JUSTIFY THE IMPLICIT STATISTICAL INDEPENDENCE ASSUMPTION HERE.  IF OKAY, JUST DELETE THIS RED BLOCK.
We assume the layer weight matrices are statistically independent, allowing us to estimate the Complexity $\mathcal{C}$, or test accuracy, with a standard Product Norm, which resembles a data dependent VC complexity
\begin{equation}
\mathcal{C}\sim\Vert\mathbf{W}_{1}\Vert\times\Vert\mathbf{W}_{2}\Vert\cdots\Vert\mathbf{W}_{L}\Vert ,
\end{equation}
where $\mathbf{W}$ is an $(N\times M)$ weight matrix, with $N\ge M$, and 
 $\Vert\mathbf{W}\Vert$ is a matrix norm.   We will actually compute the log Complexity, which takes the form 
of an Average Log Norm:
\begin{eqnarray*}
\log\mathcal{C} &\sim& \log\Vert\mathbf{W}_{1}\Vert+\log\Vert\mathbf{W}_{2}\Vert\cdots\log\Vert\mathbf{W}_{L}\Vert
\end{eqnarray*}
}
%
We have examined a large number of possible quality metrics.
The best performing metrics (recall that we can only consider metrics that do not use training/test data) depend on the norm and/or spectral properties of weight matrices, $\mathbf{W}$.%
\footnote{We do not use intra-layer information from the models in our quality metrics, but as we will describe our metrics can be used to learn about }
We consider the following metrics.
\begin{itemize}
\item 
Frobenius Norm: $\Vert\mathbf{W}\Vert^{2}_{F}=\Vert\mathbf{W}\Vert^{2}_{2}=\sum_{i,j}w^{2}_{i,j} = \sum_{i=1}^{M} \lambda_{i}^{2}$
\michael{Do we use norm or log norm, here and with other norms.  Worth being very explicit and very consistent.}
\item 
Spectral Norm: $\Vert\mathbf{W}\Vert_{\infty}=\lambda_{max}$
\michael{We should probably call the spectral norm $\Vert\mathbf{W}\Vert_{2}$, or change Frob norm notation and have an explicit remork.  Let's decide.}
\item 
Weighted Alpha Metric: $\hat{\alpha}=\alpha\log\lambda_{max}$
\item 
$\alpha$-Norm (or $\alpha$-Shatten Norm): $\Vert\mathbf{X}\Vert^{\alpha}_{\alpha}=\sum_{i=1}^{M}\lambda_{i}^{\alpha}$,
\michael{Be clear about how we average this across layers.}
\end{itemize}
The first two metrics are well-known in ML.
The last two deserve special mention.
For all these metrics, $\lambda_{i}$ is the $i^{th}$ eigenvalue of the \emph{Empirical Correlation Matrix},
$ %% $$
\mathbf{X}=\mathbf{W}^{T}\mathbf{W} ,
$ %% $$
and so $\lambda_{max}$ is the maximum eigenvalue of $\mathbf{X}$. 
(These eigenvalues are the squares of the singular values $\sigma_{i}$ of $\mathbf{W}$, i.e., $\lambda_{i}=\sigma^{2}_{i}$.)
For the last two metrics, the exponent $\alpha$ is the power law exponent that arises in the recently-developed \emph{Theory of Heavy Tailed Self Regularization (HT-SR)}~\cite{MM18_TR, MM19_HTSR_ICML, MM20_SDM}.
Operationally, $\alpha$ is determined by using the publicly-available \emph{WeightWatcher} tool (\cite{weightwatcher_package}) to fit the Empirical Spectral Density (ESD) of $\mathbf{X}$, i.e., a histogram of the eigenvalues, call it $\rho(\lambda)$, to a truncated power law
\begin{equation}
\rho(\lambda)\sim\lambda^{\alpha},\;\;\lambda\le\lambda_{max}  .
\end{equation}
\michael{We need to be clear that this is a truncated power law fit and that $\lambda_{max}$ comes from that and not the largest empirical eigenvalue.}
The Weighted Alpha Metric was introduced previously~\cite{MM20_SDM}, where (on a much smaller set of data than we consider here) it was shown to correlate well with the trends in reported test accuracies of pretrained DNNs.
Based on this, here we introduce and evaluate the $\alpha$-Norm metric.
One would expect $\hat{\alpha}$ to approximate the log $\alpha$-Norm very well for $\alpha < 2$ and reasoably well for $\alpha\in[2,5]$~\cite{MM20_unpub_work}.

\michael{Need to say that such large $\alpha$ values don't mean much.}

\michael{Need to highlight difference between $\alpha$ and $\hat{\alpha}$.}


\paragraph{Spectral Analysis of Convolutional 2D Layers.}

There is some ambiguity in performing spectral analysis on Convolutional 2D (Conv2D) layers.  
A Conv2D layer can be represented as a 4-index tensor of dimension $(w,h,in\_ch,out\_ch)$, specified by an $(w\times h)$ filter (or kernel) and $in\_ch$ / $out\_ch$ input / output channels, respectively (usually $in\_ch\le out\_ch$). 
Typically, $w=h=k$,  giving $(k\times k)$ tensor slices, or \emph{pre-Activation Maps} $\mathbf{W}_{i,L}$ of dimension $(in\_ch\times out\_ch)$ each. 
%
There are at least three different approaches that have been advocated for applying the Singular Values Decomposition (SVD) to an Conv2D layer:
run an SVD on each of the pre-Activation Maps $\mathbf{W}_{i,L}$, yielding $(k\times k)$ sets of $M$ singular values; 
stack the feature maps into a single rectangular matrix of, say, dimension $((k\times k\times out\_ch)\times in\_ch)$, yielding $in\_ch$ singular values;
compute the 2D Fourier Transform (FFT) for each of the $(in\_ch, out\_ch)$ pairs, and run SVD on the resulting Fourier coeffients~\cite{Long2019}, leading to $\sim(k\times in\_ch\times out\_ch)$ non-zero singular values.
Each method has tradeoffs.  
In principle, the third method is mathematically sound, but it is computationally expensive. 
For our analysis, because we are performing tens of thousands of calculations, we select the first method, which is numerically the fastest and is easiest to reproduce.%
\footnote{We provide a Google Colab notebook where all results can be reproduced, with the option to redo the calculations with the third option for the SVD of the Conv2D.}


\paragraph{XXX.}
XXX.  THIS VERIFICATION IS NOT ABOUT CONV LAYERS, IT IS ABOUT PL MORE GENERALLY, CORRECT?  WE SHOULD CLARIFY AND SQUISH.
To verify that our approach is meaningful, we need to confirm that the ESD is neither due to a random matrix, nor due to unsually large matrix elements, but, in fact, captures correlations learned from the data. 
We examine typical layer for the pretained AlexNet model (distributed with pyTorch). 
Figure~\ref{fig:alexnet1} displays the ESD for the first slice (or matrix $\mathbf{W}$) of the third Conv2D layer, extracted from a 4-index Tensor of shape $(384, 192, 3, 3)$.  The red line displays the best fit to a random matrix, using the Marchenko pastur theory~\cite{MM}.  We can see the random matrix model does not describe the ESD very well. For comparison, Figure \ref{fig:alexnet2} shows the ESD of the same matrix, randomly shuffled; here looks similar to the red line plot of the orginal ESD.  In fact, the empircal ESD is better modeled with a truncated power law distribtion.
\michael{We may want to give a one-sentence summary of this par and fig at the end of the previous par.}
\charles{Here, on the RMT MP stuff, I think it makes sense to point back}

\begin{figure}[H]
   \centering
   \subfigure[Actual ESD]{
     %\includegraphics[scale=0.5]{img/alexnet1.png} 
     \includegraphics[scale=0.25]{img/alexnet1.png} 
     \label{fig:alexnet1}
   }
   \subfigure[ESD of randomly shuffled matrix]{
      %\includegraphics[scale=0.5]{img/alexnet2.png}
      \includegraphics[scale=0.25]{img/alexnet2.png}
      \label{fig:alexnet2}
   }
   \caption{ESD of AlexNet Conv2D pre-Activation map for Layer 3 Slice 1, actual and randomized.
            \michael{Need better figs here.}
           }
   \label{fig:alexnet}
\end{figure}


Although the ESD is \emph{Heavy Tailed}, this does not imply that the orginal matrix $\mathbf{W}$ is itself heavy tailed--only the correlation matrix $\mathbf{X}$ is. 
If $\mathbf{W}$ was, then it would contain 1 or more unusually large matrix elements, and they would dominate the ESD.  
Of course the randomized $\mathbf{W}$ would also be heavy tailed, but its ESD neither resembles the original nor is it heavy tailed. 
So we can rule out $\mathbf{W}$ being heavy tailed.
\michael{These comments seem out of place, since they hold more generally than for the Conv2D layers.}
\charles{Agreed. We could move this up.  We have never really talked about this, but it is essential to explain the difference between assuming W is heavy tailed , which confuses everyone}

These plots tell us that the pre-activation maps of the Conv2D contains significant correlations learned from the data.  
By modeling the ESD with a power law distribution $\lambda^{\alpha}$, we can characterize the amount of correlation learned;
the smaller the exponent $\alpha$, the more correlation in the weight matrix. 
\michael{These comments seem out of place, since they hold more generally than for the Conv2D layers.}

\paragraph{Normalization of Empirical Matrices.}  
Normalization is an important, if underappreciated, practical issue.
Importantly, the normalization of weight matrices does \emph{not} affect the Power Law fits because the Heavy Tailed exponent $\alpha$ (as well as other metrics such as the Stable Rank and MP Soft Rank~\cite{MM18_TR,MM19_HTSR_ICML}) is scale-invariant.
Norm-based metrics, however, do depend strongly on the scale of the weight matrix.
Typically, to apply RMT, we would usually define Correlation Matrix with $1/N$ normalization and assume that the variance of $\mathbf{X}$ is either unity or a known constant.% 
\footnote{For formal proofs of Heavy Tailed results, one typically needs a different normalization such as $1/N^{1-\alpha}$.}
Pretrained DNNs are typically initialized with random weight matrices $\mathbf{W}_{0}$, with the variance already normalized to $\sqrt{1/N}$, or some variant of this, e.g., the Glorot/Xavier normalization~\cite{GloRot}, or a $\sqrt{2/Nk^2}$ normalization for Convolutional 2D Layers.
We do not have conrol over the final empirical normalization of these models; and we do \emph{not} normalize (or renormalize) the Empirical Correlation Matrices, i.e., we use them as-is.
The only exception to this is that we do rescale the Conv2D pre-Activation Maps $\mathbf{W}_{i,L}$ by $k/\sqrt{2}$ so that they are on the same scale as the Linear layers.

\michael{MOVE TO LATER: COMMENT ON HOW LOG NORM first and last layers behave, maybe somewhere else.}

\michael{CMOVE TO LATER: OMMENT ON HOW LOG PORM  for GPT includes unusually high alpha, not meaningful other than to show the trend.}

\paragraph{The WeightWatcher Tool.}

We compute the metrics using the WeightWatcher tool (version 0.2.7), and we provide Jupyter notebooks in the github repo for this paper~\cite{repo}, making the results fully reproducible.

\michael{XXX.  FEW SENTENCES ABOUT REPRODUCIBILITY MORE GENERALLY HERE.}



\section{Comparison of CV models}
\label{sxn:cv}

In this section, we examine empirical quality metrics described in Section~\ref{sxn:methods} for several CV model architecture series.
This includes the VGG, ResNet, and DenseNet series of models, each of which consists of several pretrained DNN models, trained on the full ImageNet~\cite{imagenet} dataset, and each of which is distributed with the current opensource pyTorch framework (version 1.4)~\cite{pyTorch}.
This also includes a larger set of ResNet models, trained on the ImageNet-1K dataset~\cite{imagenet1k}, provided on the OSMR ``Sandbox for training convolutional networks for computer vision''~\cite{osmr}, which we call the ResNet-1K series.

We perform \emph{coarse model analysis}, comparing and contrasting the four model series, and predicting trends in model quality. 
We also perform \emph{fine layer analysis}, as a function of depth for these models, illustrating that PL-based metrics can provide novel insights among the VGG, ResNet/ResNet-1K, and DenseNet architectures. 

\paragraph{Average Quality Metrics versus Reported Test Accuracies.}
We have examined the performance of the four quality metrics (Log Frobenius norm, Log Spectral norm, Weighted Alpha, and Log $\alpha$-Norm) applied to each of the VGG, ResNet, ResNet-1K, and DenseNet series.
To start, Figure~\ref{fig:vgg-metrics} considers the VGG series (in particular, the pretrained models VGG11, VGG13, VGG16, and VGG19, with and without BatchNormalization), and it plots the four quality metrics versus the reported Test accuracies~\cite{pyTorchVgg}, as well as a basic linear regression line. 
%with the Root Mean Squared Error (RMSE) shown, \michael{XXX.  WHERE IS RMSE.  IN THAT TABLE?}
All four metrics correlate quite well with the reported Top1 Accuracies, with smaller norms and smaller values of $\hat{\alpha}$ implying better generalziation (i.e., greater accuracy, lower error). 
While all four metrics perform well, notice that the Log $\alpha$-Norm metric ($\log\Vert\mathbf{W}\Vert_{\alpha}^{\alpha}$) performs best (with an RMSE of $0.42$, see Table~\ref{table:cv-models}); and the Weighted Alpha metric ($\hat\alpha =\alpha\log\lambda_{max} $), which is an approximation to the Log $\alpha$-Norm metric~\cite{MM20_unpub_work}, performs second best (with an RMSE of $0.48$, see Table~\ref{table:cv-models}).

\begin{figure}[t]
    \centering
    \subfigure[ Frobenius Norm, VGG ]{
        %\includegraphics[width=5cm]{img/VGG_lognorm_accs.png}
        \includegraphics[width=4.0cm]{img/VGG_lognorm_accs.png}
        \label{fig:vgg-fnorm}
    }
    %\qquad
    \subfigure[ Spectral Norm, VGG ]{
        %\includegraphics[width=4.9cm]{img/VGG_spectralnorm_accs.png}
        \includegraphics[width=4.0cm]{img/VGG_spectralnorm_accs.png}
        \label{fig:vgg-snorm}
    }
    %\qquad
    \subfigure[ Weighted Alpha, VGG ]{
        %\includegraphics[width=4.9cm]{img/VGG_alpha_weighted_accs.png}
        \includegraphics[width=4.0cm]{img/VGG_alpha_weighted_accs.png}
        \label{fig:vgg-walpha}
    }
    %\qquad
    \subfigure[ $\alpha$-Norm, VGG ]{
        %\includegraphics[width=4.9cm]{img/VGG_logpnorm_accs.png}
        \includegraphics[width=4.0cm]{img/VGG_logpnorm_accs.png}
        \label{fig:vgg-pnorm}
    }
    \caption{Comparison of average log Norm empirical quality metrics versus reported test accuracy for pretrained VGG models (with and without BatchNormalization), trained on ImageNet, available in pyTorch (v1.x).  Metrics fit by linear regression, RMSE reported. }
    \label{fig:vgg-metrics}
\end{figure}


\begin{figure}[t]
    \centering

    \subfigure[ ResNet, $\alpha$-Norm ]{
        %\includegraphics[width=4.2cm]{img/ResNet_logpnorm_accs.png}
        \includegraphics[width=4.0cm]{img/ResNet_logpnorm_accs.png}
        \label{fig:resnet-accuracy}
    }
    %\qquad
    \subfigure[ ResNet-1K, $\alpha$-Norm ]{
        %\includegraphics[width=4.5cm]{img/ResNet-1K_logpnorm_accs.png}
        \includegraphics[width=4.0cm]{img/ResNet-1K_logpnorm_accs.png}
        \label{fig:resnet1k-accuracy}
    }
    %\qquad
    %\subfigure[ DenseNet, $\alpha$-Norm ]{
    %    %\includegraphics[width=4.4cm]{img/DenseNet_logpnorm_accs.png}
    %    \includegraphics[width=4.0cm]{img/DenseNet_logpnorm_accs.png}
    %    \label{fig:densenet-accuracy}
    %}
    \caption{Comparison of the avergage $\alpha$-Norm empirical quality metric $\langle\log\Vert\mathbf{X}\Vert_{\alpha}^{\alpha}\rangle$ 
versus reported Top 1 reported Test Accuracy for the ResNet and ResNet-1K pretrained (pyTorch) models. }
%, and DenseNet models. }The corresponding results for VGG are shown in Figure~\ref{fig:vgg-pnorm}.  }
    \label{fig:cv2-accuracy}
\end{figure}


%MM% \begin{figure}
%MM% \centering
%MM% \begin{minipage}[b]{.22\textwidth}
%MM% %<Code for the first figure>
%MM% XXX
%MM% \caption{Caption}\label{label-a}
%MM% \end{minipage}\qquad
%MM% \begin{minipage}[b]{.22\textwidth}
%MM% %<Code for the second figure>
%MM% XXX
%MM% \caption{Caption}\label{label-b}
%MM% \end{minipage}
%MM% \end{figure}


\begin{table}[t]
\small
\begin{center}
%\begin{tabular}{|p{1in}|c|c|c|c|c|}
\begin{tabular}{|p{0.75in}|c|c|c|c|c|}
%\begin{tabular}{|l|c|c|c|c|c|}
\hline
 Series    &\#   & $\log\langle\Vert\mathbf{W}\Vert_{F}\rangle$ & $\log\langle\Vert\mathbf{W}\Vert_{\infty}\rangle$ & $\hat{\alpha}$ & $\log\langle\Vert\mathbf{X}\Vert^{\alpha}_{\alpha}\rangle$ \\
\hline
 VGG       &  6 & 0.56 & 0.53 & 0.48          & \textbf{0.42}  \\
 ResNet    &  5 & 0.9  & 1.4  & \textbf{0.61} & 0.66           \\
 ResNet-1K & 19 & 2.4  & 3.6  & \textbf{1.8}  & 1.9            \\
 DenseNet  &  4 & 0.3  & 0.26 & \textbf{0.16} & 0.21           \\
\hline
\end{tabular}
\end{center}
\caption{RMSE (smaller is better) for linear fits of quality metrics to Reported Top1 Test Error for pretrained models in each architecture series.  VGG, ResNet, and DenseNet were pretrained on ImageNet, and ResNet-1K was pretrained on ImageNet-1K. 
}
\label{table:cv-models}
\end{table}

See Table~\ref{table:cv-models} for a summary of results for Top1 Accuracies for all four metrics on for the VGG, ResNet, and DenseNet series.
Similar results (not shown) are obtained for the Top5 Accuracies.
Overall, for the the ResNet, ResNet-1K, and DenseNet series, all metrics perform relatively well, the Log $\alpha$-Norm metric performs second best, and th\
e Weighted Alpha metric performs best.
These model series are all well-trodden, and our results indicate that norm-based metrics and PL-based metrics can both distinguish among ``good-better-bes\
t'' models, with PL-based metrics performing somewhat better.

Notice that the DenseNet series,  only has 4 data points.  In our larger analysis, in Section~\ref{sxn:all} we only include series with 5 or more models.
For completemnss and reproducibility here, the DenseNet Log $\alpha$-Norm plot is given in the Appendix.

\paragraph{Variations in Data Set Size}
We want to point out, in particular, how the empirical log Norm metrics vary with data set size.
Figure~\ref{fig:cv2-accuracy} plots and compares the 
Log $\alpha$-Norm
metric--the best performing metric--for the full ResNet, trained on the full ImageNet dataset, against the ResNet-1K, which has been trained on a much smaller ImageNet-1K data set.
%(A much more detailed set of plots, including the behavior of all four methods on each of the series, is available at \cite{XXX-WEB-LINK}.)
Here, the Log $\alpha$-Norm is much better that the Log Frobenius/Spectral norm metrics (although, as Table~\ref{table:cv-models} shows, it is actually slightly worse than the Weighted Alpha metric).
The ResNet series, has an RMSE of $0.66$, whereas the ResNet-1K series also shows good correlation, but has a much larger RMSE of $1.9$.
As expected, the higher quality data set shows a better fit, even with less data points.

\paragraph{Correlation Flow}
We can learn much more about a pretrained model by going beyond average values to examining quality metrics for each layer weight matrix, $\mathbf{W}$, as a function of depth (or layer id).  % in the network. 
%The most interesting results are seen when we 
For example, we can 
plot (just) the PL exponent, $\alpha$,%
\footnote{That is, here we consider just $\alpha$ for each layer, i.e., not $\hat{\alpha}$ or $\Vert\mathbf{W}\Vert^{\alpha}_{\alpha}$.} 
as a function of depth.
%
See Figure~\ref{fig:3models-alpha-layers}, which plots $\alpha$ for each layer (the first layer corresponds to data, the last layer to labels) for the least accurate (shallowest) and most accurate (deepest) model in each of the VGG (without BatchNormalization), ResNet-1K, and DenseNet series.
(Again, much more detailed set of plots is available at \cite{XXX-WEB-LINK}; but note that the corresponding layer-wise plots for Frobenius and Spectral norms are much less interesting than the results we present here.)

\begin{figure}[t]
    \centering

    \subfigure[ VGG ]{
        %\includegraphics[width=4.5cm]{img/VGG_fnl_alpha_depth.png} 
        \includegraphics[width=4.0cm]{img/VGG_fnl_alpha_depth.png} 
                \label{fig:vgg-alpha-layers}
    }
    %\qquad
    \subfigure[ ResNet ]{
        %\includegraphics[width=4.5cm]{img/ResNet_fnl_alpha_depth.png} 
        \includegraphics[width=4.0cm]{img/ResNet_fnl_alpha_depth.png} 
        \label{fig:resnet-alpha-layer}
    }
    %\qquad
    \subfigure[ DenseNet ]{
        %\includegraphics[width=4.5cm]{img/DenseNet_fnl_alpha_depth.png} 
        \includegraphics[width=4.0cm]{img/DenseNet_fnl_alpha_depth.png} 
        \label{fig:densenet-alpha-layer}
    
}    \subfigure[ ResNet (overlaid) ]{
        \includegraphics[width=4.0cm]{img/resnet_alpha_overlaid_depth.png} 
        \label{fig:resnet_alpha_overlaid_depth}
    }
    \caption{Correlation Flow: Power Law exponent $\alpha$ versus layer id, for the least and the most accurate
      models in VGG (a), ResNet (b), and DenseNet (c) series (VGG without BatchNorm, and the Y axes on each plot are different.)  
             Figure (d) displays the ResNet models (b), zoomed in to $alpha\in[1,5]$, and with the layer ids overlaid on the X-axis, to
             allow a more detailed analysis for the most strongly correlated layers.
             Notice that ResNet152 exhibits different and much more stable behavior across layers
             This contrasts with how both VGG models gradually worsen in deeper layers and how the DenseNet models are much more erratic.  
            }
    \label{fig:3models-alpha-layers}
\end{figure}

In the VGG models, Figure~\ref{fig:vgg-alpha-layers} shows that the PL exponent $\alpha$ systematically increases as we move down the network, from data to labels, in the Conv2D layers, starting with $\alpha\lesssim 2.0$ and reaching all the way to $\alpha\sim 5.0$; and then, in the last three, large, fully-connected (FC) layers, $\alpha$ stabilizes back down to $\alpha\in[2,2.5]$.
This is seen for all the VGG models (again, only the shallowest and deepest are shown in this figure), indicating that the main effect of increasing depth is to increase the range over which $\alpha$ increases, thus leading to larger $\alpha$ values in later Conv2D layers of the VGG models.
This is quite different than the behavior of either the ResNet-1K models or the DenseNet models.

For the ResNet-1K models, Figure~\ref{fig:resnet-alpha-layer} shows that $\alpha$ also increases in the last few layers (more dramatically, in fact, observe the differing scales on the Y axes).
However, as the ResNet-1K models get deeper, there is a wide range over which $\alpha$ values tend to remain small.
%\michael{Can we say something about exceptions which are larger than for VGG or DenseNet.}
%\charles{You have the same plots I do..what do you want to say >}
%\michael{Are they small or different types of layers or something?}
%\charles{Not that I am aware of.  Ill think on it}
This is seen for other models in the ResNet-1K series, but it is most pronounced for the larger ResNet-1K (152) model, where $\alpha$ remains relatively stable at $\alpha\sim 2.0$, from the earliest layers all the way until we reach close to the final layers.  

For the DenseNet models, Figure~\ref{fig:densenet-alpha-layer} shows that $\alpha$ tends to increase as the layer id increases, in particular for layers toward the end.
While this is similar to what is seen in the VGG models, with the DenseNet models, $\alpha$ values increase almost immediately after the first few layers, and the variance is much larger (in particular for the earlier and middle layers, where it can range all the way to $\alpha\lesssim 8.0$) and much less systematic throughout the network.

%Recall the differences---in particular, the number of residual connections---between the VGG, ResNet, and DenseNet architectures.
%VGG, while a fine model in many ways, was limited by needing massive FC layers at the end of the model, requiring significant memory and computational resources, and it's poor conditioning led to problems with vanishing gradients.  
%The empirical manifestations of this (on weight matrices) are that fitted $\alpha$ values get much worse for deeper layers.
\ncyan{We can interpret these observations by recalling the architectural differences between the VGG, ResNet, and DenseNet architectures,
and, in particular, the number of of residual connections. VGG resembles the  traditional convolutional architectures,
such as LeNet5, and consists of several [Conv2D-Maxpool-ReLu] blocks, followed by 3 large Fully Connected (FC) layers.
ResNet greatly improved on VGG by replacing the large FC layers, shrinking the Conv2D blocks, and introducing \emph{residual connections}.
This optimized approach allows for greater accurary with far fewer parameters (and GPU memory requirements), and
ResNet models of up to 1000 layers have been trained.\cite{resnet1000}.}
We conjuecture that the effeciency and effectivness of ResNet is reflected in the smaller and more stable $\alpha\sim 2.0$, across nearly all layers, indicating that the inner layers are very well correlated and strongly optimized.
Contrast this with the DenseNet models, which contains many connections between every layer.
Our results (large $\alpha$, meaning they even a HT model is probably a poor fit) suggest that DenseNet has too many connections, diluting high quality interactions across layers, and leaving many layers very poorly optimized.

More generally, we can understand Figure~\ref{fig:vgg-alpha-layers} in terms of the \emph{Correlation Flow}. 
Recall that the log $\alpha$-Norm metric and the Weighted Alpha metric are based on HT-SR Theory~\cite{MM18_TR, MM19_HTSR_ICML, MM20_SDM}, which is in turn based on ideas from the statisical mechanics of heavy tailed and strongly correlated systems~\cite{BouchaudPotters03, SornetteBook, BP11, bun2017}. 
There, one expects the weight matrices of well-trained DNNs will 
exhibit correlations over many size scales. Their ESDs can be well-fit by a (truncated) Power law (PL), with exponents $\alpha\in[2,4]$.
Much larger values $(\alpha\gg 5\;or\;6)$ may relfect poorer PL fits, whereas smaller values $(\alpha\rightarrow 2)$, are associated with models
that generalize better.
%, but the presence of many adjacent layers with relatively good PL fits \emph{suggests} that those adjacent layers ``impedance match'' well and that the correlations that are captured by one layer propagate well to subsequent layers.
Informally, one would expect a DNN model to perform well when it fascilitates the propagation of information/features across layers.
Previous studies argue this by computing the gradients over the input data.
In the absence of training/test data, we can to try to quantify this by measuring the PL properties of empirical weight matrices.
Smaller $\alpha$ values correspond to layers in which correlations across multiple scales are better captured~\cite{MM18_TR,SornetteBook}, and thus we expect that small values of $\alpha$ that are stable across multiple layers enable better \emph{correlation flow} through the network.


%\paragraph{Layer Analysis: Behavior on Less Thoroughy Studied CV Models.}
\paragraph{Correlation Flow: PL vs. Norm Metrics.}

The similarity between norm-based metrics and $\alpha$-based metrics suggests a question: is the Weighted Alpha metric just a variation of the more familiar norm-based metrics, or does it capture something different?  
More generally, do fitted $\alpha$ values contain information not captured by norms? 
To show that it is not just a variation and that $\alpha$ does capture something different, consider the following example, which looks at a compressed / distilled DNN model~\cite{CWZZ17_TR}.


\begin{figure}[t]
   \centering
   \subfigure[$\lambda_{max}$ for ResNet20 layers]{
      %\includegraphics[scale=0.14]{img/resnet4d_maxev.png}
      \includegraphics[width=4.0cm]{img/resnet4d_maxev.png}
      %\includegraphics[width=4.5cm]{img/resnet4d_maxev.png}
      \label{fig:resnet204Dmaxev}
   }
   %\qquad
   \subfigure[$\alpha$ for ResNet20 layers]{
      %\includegraphics[scale=0.14]{img/resnet4d_alphas.png}
      \includegraphics[width=4.0cm]{img/resnet4d_alphas.png}
      %\includegraphics[width=4.5cm]{img/resnet4d_alphas.png}
      \label{fig:resnet204Dalpha}
   }
   \caption{Correlation Flow.
            ResNet20, distilled with Group Regularization, as implemented in the \texttt{distiller} (4D\_regularized\_5Lremoved) pre-trained models.  
            Spectral Norm, $\log\lambda_{max}$, and, PL exponent, $\alpha$, for individual layer, versus layer id, for both baseline (before distillation, green) and finetuned (after distillation, red) pre-trained models. 
           }
   \label{fig:resnet204D5L}
\end{figure}

\paragraph{Scale Collapse; or, Distillation may break models}

In examining hundreds of pretrained models, we have found several anamolies that demonsrate the power of our approach.
Here, we demonstrate that some distillations methods may actually \emph{break} models unexpectedly by introducing what we call \emph{Scale Collapse},
where several distilled layers have unexpectedly small Spectral Norms.
We consider ResNet20, trained on CIFAR10, before and after applying the Group Regularization distillation technique, as implemented in the \texttt{distiller} package.%
\footnote{For details, see \url{https://nervanasystems.github.io/distiller/\#distiller-documentation} and also \url{https://github.com/NervanaSystems/distiller}.}
We analyze the pre-trained 4D\_regularized\_5Lremoved baseline and finetuned models.  %\cite{distiller_repo}.
The reported baseline test accuracies ($Top1=91.45$ and $Top5=99.75$) are better than the reported finetuned test accuracies ($Top1=91.02$ and $Top5=99.67$)~\cite{XXX-XXX}.  Because the baseline accuracy is greater,  the previous results on ResNet (Table~\ref{table:cv-models} and Figure~\ref{fig:cv2-accuracy}) suggests that the baseline Spectral Norms should be \emph{smaller} on average than the finetuned ones should be smaller.
\emph{The oppopsite is observed.}
%\michael{Do we have something backwards here: where we observed that the average norm to increase with decreasing test error (as it ``should'' not), whereas the average PL exponent $\alpha$ decreases (as ``expected'' from HT-SR Theory).  }
%In both cases (Frobenius norm results not shown), we observe the opposite.
Figure~\ref{fig:resnet204D5L} presents the Spectral Norm (here denoteds $\log\lambda_{max}$) and PL exponent ($\alpha$) for each individual layer weight matrix $\mathbf{W}$.%
\footnote{Here, we only include layer matrices or feature maps with $M\ge50$.}
On the other hand, the $\alpha$ values do not differ systematically between the baseline and finetuned models.
%\michael{Some comment about big jump in norm; but fix previous backwards question first.}--SEE BELOW
Also (not shown), the average (unweighted) baseline $\alpha$ is \emph{smaller} than the finetuned average (as predicted by HT-SR Theory, the basis of $\hat{\alpha}$).

Note that Figure ~\ref{fig:resnet204Dalph} depcits 2 very large $\alpha\gg 6$ for the baseline model, but not for the finetuned model.
This suggests that the baseline model has at least 2 over-parameterized layers, and the distillation method does, in fact, 
improve the finetuned model by compressing these layers.s

\michael{XXX.  WORK ON THIS PAR AFTER DO NLP EXAMPLES.}
Our interpretation of this is the following.
The pre-trained models in the \texttt{distiller} package have passed some quality metric, but they are much less well trodden than any of the 
VGG, ResNet, or DenseNet series we cosidered above.
\charles{MOvE THIS SOMEWHERE ELSE AND CLEAN UP THIS SECTION Notice while norms make good regularizers for a single model, there is no reason \emph{a priori} 
to expect them correlate so well with the test accuracies across different models.
However, we do expect the PL $\alpha$ to do so because it effectively measures the amount of correlation in the model}
%
The reason for this is that the \texttt{distiller} Group Regularization technique 
%%has the unusual effect of increasing
\nred{spuriously increases}
 the norms of the $\mathbf{W}$ pre-activation maps for at least two of the Conv2D layers.
\michael{This impedance shuffling or whatever it is called messes up norms, but the strong correlations are still preserved, as seen in the $\hat{\alpha}$ metric, and the model quality is good.  CLARIFY.}
\michael{XXX.  INTERPRET IN TERMS OF XXX GOOD VERSUS BAD, AS OPPOSED TO GOOD BETTER BEST.}



\section{Comparison of NLP Transformer Models}
\label{sxn:nlp}

\paragraph{GPT vs GPT2}

\paragraph{Power Law Exponents}

\begin{figure}
   \includegraphics[scale=0.25]{img/gpt-alpha-hist.png}
   \caption{Comparison of heavy tailed power law exponent $\alpha$ for OpenAI GPT and GPT pretrained models}

   \label{fig:gpt-alphs-hist}
\end{figure}


\begin{figure}[t]
    \centering

    \subfigure[ Power Law Exponent $\alpha$  ]{
        \includegraphics[width=4.5cm]{img/gpt2-alpha-hist.png}
        \label{fig:gpt-alpha-layer}
    }
    \qquad
    \subfigure[ Spectral Norm $\Vert\mathbf{W}\Vert_{\infty}$]{
        \includegraphics[width=4.5cm]{img/gpt2-snorm-hist.png}
        \label{fig:resnet-snorm-layer}
    }
    \qquad
    \subfigure[ Alpha-Norm $\Vert\mathbf{X}\Vert_{\alpha}^{\alpha}$]{
        \includegraphics[width=4.5cm]{img/gpt2-pnorm-hist.png}
        \label{fig:gpt-pnorm-layer}
    }
    \caption{Comparison of Power Law Exponnents, Spectral Norm, and Alpha-Norm for different size models in the GPT2 architecture series.    }
    \label{fig:gpt-alpha-layers}
\end{figure}


\paragraph{Correlation Flow}

\begin{figure}[t]
    \centering

    \subfigure[ Power Law Exponent $\alpha$  ]{
        \includegraphics[width=4.0cm]{img/gpt-alpha-layer.png}
        \label{fig:gpt-alpha-layer}
    }
    \qquad
    \subfigure[ Spectral Norm $\Vert\mathbf{W}\Vert_{\infty}$]{
        \includegraphics[width=4.5cm]{img/gpt-snorm-layer.png}
        \label{fig:resnet-snorm-layer}
    }
    \qquad
    \subfigure[ Alpha-Norm $\Vert\mathbf{X}\Vert_{\alpha}^{\alpha}$]{
        \includegraphics[width=4.5cm]{img/gpt-pnorm-layer.png}
        \label{fig:gpt-pnorm-layer}
    }
    \caption{Comparison of Correlation Flow and Spectral Norm for OpenAI GPT and GPT2   }
    \label{fig:gpt-alpha-layers}
\end{figure}

\paragraph{GPT2: small, medium, large, extra-large}




\section{Comparison of all pretrained CV Models}
\label{sxn:all_cv_models}


\section{Conclusion}
\label{sxn:conc}



We should point out that it is not obvious that norm-based metrics will perform well \emph{across} models, as they are usually used \emph{within} a given model or parameterized model class.
That PL-based metrics perform better should not be surprising---while norms are very coarse metrics of the properties of a matrix, 

That being said, our goal is not to make a statement about every publicly-available model.
Ther are exceptions to the general trends we state.
Several of these are described below, and others are are available in what we release.

\michael{MENTION:} 
Have some sort of caveat about how we don't solve every problem in the world, in particular there are exceptions to ghe general trends we note, but that all of our work is publicly available in Weightwatcher and Google Colab, so people can reproduce.
\michael{XXX.  FEW SENTENCES ABOUT REPRODUCIBILITY MORE GENERALLY HERE, HERE OR IN DISCUSSION/CONCLUSION.}

\michael{MENTION:} 
PL-based metrics have been used to characterize the degree of strong correlations in a system, and they are derived from the recently-developed Theory of Heavy-Tailed Self Regularization (HT-SR).
AND DEFINE HT-SR ACRONYM.





\michael{
Where, if anywhere to put the following:
\begin{itemize}
\item MOVE TO LATER: COMMENT ON HOW LOG NORM first and last layers behave, maybe somewhere else.
\item MOVE TO LATER: COMMENT ON HOW LOG PORM  for GPT includes unusually high alpha, not meaningful other than to show the trend.
\end{itemize}
}

XXX.  THE FOLLOWING IS PROBABLY BETTER HERE.
Distinguish between what we will call a
\emph{phenomenological theory}
(that describes empirical relationship of phenomena to each other, in a way which is consistent with fundamental theory, but is not directly derived from that theory)
and what can be called a 
\emph{first principles theory} 
(that is applicable to tiny things but has no hope of scaling up).
\charles{it is the oppposite...ab initio theory scales quite well...it is the spherical cow models of physics and ML that do not scale. What we have is a semi-empirical theory.  We use real theory, but require empirical input, at least in the new stat mech work.  What we have introduced is a phenomenology, which to me is different from a semi-empirical or phenomenological theory  }
\michael{Let's touch base to get this right.}
\footnote{In most areas where there are complex highly-engineered systems (beyond complex AI/ML systems), one used phenomenological theory rather than first principles theory.  For example, one does not try to solve the Schr\"odinger equation if one is interested in building a bridge or an airplane.}




XXX.  PUT CONCLUSION HERE AND WEAVE IN COMMENTS FROM BELOW.


Some other comments that we need to weave into a narrative eventually after later sections are written:
\begin{itemize}
\item
GPT versus GPT2.
What happens when we don't have enough data?
This is the main question, and we can use out metrics to evaluate that, but we also get very different results for GPT versus GPT2.
\item
The spectral norm is a regularizer, used to distinguish good-better-best, not a quality metric.
For example, it can ``collapse,'' and for bad models we can have small spectral norm.
So, it isn't really a quality metric.
\item
One question that isn't obvious is whether regularization metrics can be used as quality metrics.
One might think so, but the answer isn't obviously yes.
We show that the answer is No.
A regularizer is designed to select a unique solution from a non-unique good-better-best.
Quality metrics can also distinguish good versus bad.
\item
(We should at least mention this is like the statistical thing where we evaluate which model is better, as oposed to asking if a given model is good, I forget the name of that.)
\item
There are cases where the model is bad but regularization metric doesn't tell you that.
Quality should be correlated in an empirical way.
Correlated with good-better-best; but also tell good-bad.
\item
Question: why not use regularier for quality?
Answer: A regularizer selects from a given set of degenerate models one which is nice or unique.
It doesn't tell good versus bad, i.e., whether that model class is any good.
\item
Thus, it isn't obvious that norm-based metrics should do well, and they don't in general.
\item
We give examples of all of these: bad data; defective data; and distill models in a bad way.
(Of course, bad data means bad model, at least indirectly, since the quality of the data affects the properties of the model.)
\item
We can select a model and change it, i.e., we don't just do hyperparameter fiddling.
\end{itemize}

%Perhaps our most improtant contribution, however, is just doing things different, etc. ...


\vspace{-2mm}
\noindent
\paragraph{Acknowledgements.}
MWM would like to acknowledge ARO, DARPA, NSF, and ONR as well as the UC Berkeley BDD project and a gift from Intel for providing partial support of this work.
We would also like to thank Amir Khosrowshahi and colleagues at Intel for helpful discussion regarding the Group Regularization distillation technique.

   %% \bibliographystyle{unsrt}
   %% {\small
   %% \bibliography{dnns}
   %% }

{\small

\begin{thebibliography}{10}

\bibitem{weightwatcher_package}
{WeightWatcher}, 2018.
\newblock \url{https://pypi.org/project/WeightWatcher/}.

\bibitem{kdd20_sub_repo}
\url{https://github.com/CalculatedContent/ww-trends-2020}.

\bibitem{EB01_BOOK}
A.~Engel and C.~P. L.~Van den Broeck.
\newblock {\em Statistical mechanics of learning}.
\newblock Cambridge University Press, New York, NY, USA, 2001.

\bibitem{MM17_TR}
C.~H. Martin and M.~W. Mahoney.
\newblock Rethinking generalization requires revisiting old ideas: statistical
  mechanics approaches and complex learning behavior.
\newblock Technical Report Preprint: arXiv:1710.09553, 2017.

\bibitem{BKPx20}
Y.~Bahri, J.~Kadmon, J.~Pennington, S.~Schoenholz, J.~Sohl-Dickstein, and
  S.~Ganguli.
\newblock Statistical mechanics of deep learning.
\newblock {\em Annual Review of Condensed Matter Physics}, pages 000--000,
  2020.

\bibitem{MM18_TR}
C.~H. Martin and M.~W. Mahoney.
\newblock Implicit self-regularization in deep neural networks: Evidence from
  random matrix theory and implications for learning.
\newblock Technical Report Preprint: arXiv:1810.01075, 2018.

\bibitem{MM19_HTSR_ICML}
C.~H. Martin and M.~W. Mahoney.
\newblock Traditional and heavy-tailed self regularization in neural network
  models.
\newblock In {\em Proceedings of the 36th International Conference on Machine
  Learning}, pages 4284--4293, 2019.

\bibitem{MM19_KDD}
C.~H. Martin and M.~W. Mahoney.
\newblock Statistical mechanics methods for discovering knowledge from modern
  production quality neural networks.
\newblock In {\em Proceedings of the 25th Annual ACM SIGKDD Conference}, pages
  3239--3240, 2019.

\bibitem{MM20_SDM}
C.~H. Martin and M.~W. Mahoney.
\newblock Heavy-tailed {U}niversality predicts trends in test accuracies for
  very large pre-trained deep neural networks.
\newblock In {\em Proceedings of the 20th SIAM International Conference on Data
  Mining}, 2020.

\bibitem{BouchaudPotters03}
J.~P. Bouchaud and M.~Potters.
\newblock {\em Theory of Financial Risk and Derivative Pricing: From
  Statistical Physics to Risk Management}.
\newblock Cambridge University Press, 2003.

\bibitem{SornetteBook}
D.~Sornette.
\newblock {\em Critical phenomena in natural sciences: chaos, fractals,
  selforganization and disorder: concepts and tools}.
\newblock Springer-Verlag, Berlin, 2006.

\bibitem{BP11}
J.~P. Bouchaud and M.~Potters.
\newblock Financial applications of random matrix theory: a short review.
\newblock In G.~Akemann, J.~Baik, and P.~Di Francesco, editors, {\em The Oxford
  Handbook of Random Matrix Theory}. Oxford University Press, 2011.

\bibitem{bun2017}
J.~Bun, J.-P. Bouchaud, and M.~Potters.
\newblock Cleaning large correlation matrices: tools from random matrix theory.
\newblock {\em Physics Reports}, 666:1--109, 2017.

\bibitem{NTS15}
B.~Neyshabur, R.~Tomioka, and N.~Srebro.
\newblock Norm-based capacity control in neural networks.
\newblock In {\em Proceedings of the 28th Annual Conference on Learning
  Theory}, pages 1376--1401, 2015.

\bibitem{BFT17_TR}
P.~Bartlett, D.~J. Foster, and M.~Telgarsky.
\newblock Spectrally-normalized margin bounds for neural networks.
\newblock Technical Report Preprint: arXiv:1706.08498, 2017.

\bibitem{LMBx18_TR}
Q.~Liao, B.~Miranda, A.~Banburski, J.~Hidary, and T.~Poggio.
\newblock A surprising linear relationship predicts test performance in deep
  networks.
\newblock Technical Report Preprint: arXiv:1807.09659, 2018.

\bibitem{MM20_unpub_work}
C.~H. Martin and M.~W. Mahoney.
\newblock Unpublished results, 2020.

\bibitem{imagenet}
O.~Russakovsky et~al.
\newblock Imagenet large scale visual recognition challenge.
\newblock {\em International Journal of Computer Vision}, 115(3):211--252,
  2015.

\bibitem{pytorch}
A.~Paszke et~al.
\newblock Pytorch: An imperative style, high-performance deep learning library.
\newblock In {\em Annual Advances in Neural Information Processing Systems 32:
  Proceedings of the 2019 Conference}, pages 8024--8035, 2019.

\bibitem{osmr}
{Sandbox for training convolutional networks for computer vision}.
\newblock \url{https://github.com/osmr/imgclsmob}.

\bibitem{resnet1000}
K.~He, X.~Zhang, S.~Ren, and J.~Sun.
\newblock Identity mappings in deep residual networks.
\newblock Technical Report Preprint: arXiv:1603.05027, 2016.

\bibitem{CWZZ17_TR}
Y.~Cheng, D.~Wang, P.~Zhou, and T.~Zhang.
\newblock A survey of model compression and acceleration for deep neural
  networks.
\newblock Technical Report Preprint: arXiv:1710.09282, 2017.

\bibitem{distiller}
{Intel Distiller package}.
\newblock \url{https://nervanasystems.github.io/distiller}.

\bibitem{Attn2017}
A.~Vaswani et~al.
\newblock Attention is all you need.
\newblock Technical Report Preprint: arXiv:1706.03762, 2017.

\bibitem{huggingface}
T.~Wolf et~al.
\newblock Huggingface's transformers: State-of-the-art natural language
  processing.
\newblock Technical Report Preprint: arXiv:1910.03771, 2019.

\bibitem{gpt2-xl}
{OpenAI GPT-2: 1.5B Release}.
\newblock \url{https://openai.com/blog/gpt-2-1-5b-release/}.

\bibitem{CNNSVD}
H.~Sedghi, V.~Gupta, and P.~M. Long.
\newblock The singular values of convolutional layers.
\newblock Technical Report Preprint: arXiv:1805.10408, 2018.

\bibitem{GloBen10}
X.~Glorot and Y.~Bengio.
\newblock Understanding the difficulty of training deep feedforward neural
  networks.
\newblock In {\em Proceedings of the 13th International Workshop on Artificial
  Intelligence and Statistics}, pages 249--256, 2010.

\end{thebibliography}

}

\appendix
\section{Appendix}
\label{sxn:appendix}

In this appendix, we provide more details on several issues that are important for reproducibility of our results.

\subsection{Reproducibility Considerations}


\paragraph{SVD of Convolutional 2D Layers.}

There is some ambiguity in performing spectral analysis on Conv2D layers.  
Each layer is a 4-index tensor of dimension $(w,h,in,out)$, with an $(w\times h)$ filter (or kernel) and $(in, out)$
channels. When $w=h=k$,  giving $(k\times k)$ tensor slices, or \emph{pre-Activation Maps} $\mathbf{W}_{i,L}$ of dimension $(in\times out)$ each. 
%
We identify 3 different approaches for running SVD on a Conv2D layer:
\begin{enumerate}
\item run SVD on each pre-Activation Map $\mathbf{W}_{i,L}$, yielding $(k\times k)$ sets of $M$ singular values
\item stack the maps into a single matrix of, say, dimension $((k\times k\times out)\times in)$, run SVD to get $in$ singular values
\item compute the 2D Fourier Transform (FFT) for each of the $(in, out)$ pairs, and run SVD on the Fourier coeffients~\cite{Long2019}, leading to $\sim(k\times in\times out)$ non-zero singular values.
\end{enumerate}
Each method has tradeoffs.  
Method (3) is mathematically sound, but computationally expensive. Method (2) is ambiguous.
For our analysis, because we need thousands of runs, we select method (1), which is the fastest (and is easiest to reproduce).

\paragraph{Normalization of Empirical Matrices.}  
Normalization is an important, if underappreciated, practical issue.
Importantly, the normalization of weight matrices does \emph{not} affect the PL fits because $\alpha$ is scale-invariant.
Norm-based metrics, however, do depend strongly on the scale of the weight matrix--\nred{that is the point.}
%\nred{Indeed, early theoretical work by Bartlett suggests that the test accuracy depends strongly on the ``total size'' of the weight matrics.}
To apply RMT, we usually define $\mathbf{X}$ with $1/N$ normalization, assuming variance of $\sigma^{2}=1.0$.
%\footnote{For Heavy Tailed theorems, one typically needs a normalization such as \nred{$1/N^{\alpha-1}$. check this}}
Pretrained DNNs are typically initialized with random weight matrices $\mathbf{W}_{0}$, with
 $\sigma^{2}\sim 1/\sqrt{N}$, or some variant, e.g., the Glorot/Xavier normalization~\cite{GloRot}, or a $\sqrt{2/Nk^2}$ normalization for Convolutional 2D Layers. With this implicit scale, 
we do \emph{not} ``renormalize'' the empirical weight matrices, i.e., we use them as-is.
The only exception is that we do rescale the Conv2D pre-activation maps $\mathbf{W}_{i,L}$ 
by $k/\sqrt{2}$ so that they are on the same scale as the Linear / Fully Connected (FC) layers.

\paragraph{Special consideration for NLP models.}
NLP models, and other models with large initial embeddings require special care because the
embedding layers frequently lack the implicit $latex\frac{1}{\sqrt{N}}$ normalization present in other layers.
For example, in GPT, most layers, the maximum eigenvalue $\lambda_{max}\sim\mathcal{O}(10-100)$,
but in the first embedding layer, the maximum is of order N (the number of words in the embedding), or
 $\lambda_{max}\sim\mathcal{O}(10^{5})$.  For GPT and GPT2, we treat all layers as-is (although one may to normalize
the first 2 layers by  $\mathbf{X}$ by $\frac{1}{N}$, or to treat them as an outlier).

\subsection{Reproducing Sections 4 and 5}

We provide a github repository for this paper that includes jupyter notsbooks that fully reproduce all results.
All results have been produced using the weightwatcher tool (v0.2.7).
The ImageNet and OpenAI GPT pretrained models are provided in the current pyTorch, torchvision, and huggingface distributions,
as specified in the \texttt{requirements.txt} file.

\begin{table}[t]
\small
\begin{center}
\begin{tabular}{|p{1in}|c|}
\hline
Figure & Jupyter Notebook \\
\hline
1  &  WeightWatcher-VGG.ipynb \\
2(a)  &  WeightWatcher-ResNet.ipynb \\
2(b)  &  WeightWatcher-ResNet-1K.ipynb \\
3(a)  &  WeightWatcher-VGG.ipynb \\
3(b)  &  WeightWatcher-ResNet.ipynb \\
3(c)  &  WeightWatcher-DenseNet.ipynb \\
\hline
5 & WeightWatcher-OpenAI-GPT.ipynb \\
6, 7 & WeightWatcher-OpenAI-GPT2.ipynb \\
\hline
\end{tabular}
\end{center}
\caption{Jupyter notebooks used to reproduce Tables 1 and 2, and Figures 1,2,3,5,6, and 7.}
\label{table:notebooks}
\end{table}

\subsection{Reproducing Section 3, Distiller Model}


\subsection{Reproducing Table 3, Section 6}

We provide several Google Colab notebooks which can be used to reproduce the result in Table 3.
The ImageNet-1K and other pretrained models are taken from the pytorch models in the \texttt{omsr/imgclsmob} 
``Sandbox for training convolutional networks for computer vision'' github repository.
\footnote{\url{https://github.com/osmr/imgclsmob}}
The data can be generated in parallel by running each Colab notebook simultaneously on the same account,
The final results are generated with \charles{move notebooks to repo and finish}

We attempt to run linear regressions for all pyTorch models for each architecture series for all datasets provided.  
There are over $450$ models in all, and we note that the \texttt{osmr/imgclsmob} repository is constantly being updated with new models.
We omit the results for CUB-200-2011, Pascal-VOC2012, ADE20K, and COCO datasets as there are less than 15 models
for those datasets. The final datasets used are shown in Table~\ref{table:datasets}
The final architecture series used are shown in  Table~\ref{table:architectures}, with the number of models in each.

To further explain how to reproduce our analysis, we run three batches of linear regressions. First at the global level, we divide models by datasets and run regression separately on all models of a certain dataset, regardless of the architecture. At this level, the plots are quite noisy and clustered as each architecture has its own accuracy trend but, you could still see that most plots show positive relationship with positive coefficients
The regressions are shown in Figure~\ref{fig:DSalphas}.

To generate the results in Table 3, we run linear for each architecture series in Table~\ref{table:architectures},
regressing each empirical log norm metric against the reported Top 1 (and Top 5) errors (as listed on the \texttt{osmr/imgclsmob} github 
repository README file. 
We filter out regressions with less than five datapoints.
We record the R-squared and mean squared errors (MSE). 
The final results for all models is provided in the \texttt{XXXXXX}.

\begin{table}[t]
\small
\begin{center}
\begin{tabular}{|p{1in}|c|}
\hline
Dataset & $\#$ of Models \\
\hline
imagenet-1k   &  78 \\
svhn          &  30 \\
cifar-100     &  30 \\
cifar-10      &  18 \\
cub-200-2011  &  12 \\
\hline
\end{tabular}
\end{center}
\caption{Datasets used}
\label{table:datasets}
\end{table}

\begin{table}[t]
\small
\begin{center}
\begin{tabular}{|p{2in}|c|}
\hline
Architecture & $\#$ of Models \\
\hline
ResNet                                     & 30 \\
SENet/SE-ResNet/SE-PreResNet/SE-ResNeXt    & 24 \\
DIA-ResNet/DIA-PreResNet                   & 18 \\
ResNeXt                                    & 12 \\
WRN                                        & 12 \\
DLA                                        & 6 \\
PreResNet                                  & 6 \\
ProxylessNAS                               & 6 \\
VGG/BN-VGG                                 & 6 \\
IGCV3                                      & 6 \\
EfficientNet                               & 6 \\
SqueezeNext/SqNxt                          & 6 \\
ShuffleNet                                 & 6 \\
DRN-C/DRN-D                                & 6 \\
ESPNetv2                                   & 6 \\
HRNet                                      & 6 \\
SqueezeNet/SqueezeResNet                   & 6 \\
\hline
\end{tabular}
\end{center}
\caption{Architectures used}
\label{table:architectures}
\end{table}


\begin{figure}[t]
    \centering
    \subfigure[ImageNet 1K]{
        \includegraphics[width=2.5cm]{img/imagenet1k_alpha.png}
        \label{fig:imagenet1k-alpha}
    }
    \qquad
    \subfigure[ CIFAR 10 ]{
        \includegraphics[width=2.5cm]{img/cifar10_alpha.png}
        \label{fig:cifar10.alpha}
    }
    \qquad
    \subfigure[ CIFAR 100 ]{
        \includegraphics[width=2.5cm]{img/cifar100_alpha.png}
        \label{fig:cifar100.alpha}
    }
    \qquad
    \subfigure[ SVHN ]{
        \includegraphics[width=2.5cm]{img/svhn_alpha.png}
        \label{fig:svhn.alpha}
    }
    \qquad
    \subfigure[ CUB 200 ]{
        \includegraphics[width=2.5cm]{img/cub200_alpha.png}
        \label{fig:cub200.alpha}
    }
    \caption{\charles{Preliminary charts:} PL exponent $\alpha$ vs. reported Top1 Test Accuracies for pretrained DNNs available\charles{ref} for 5 different data sets.}

    \label{fig:DSalphas}
\end{figure}





\subsection{XXX: PLACEHOLDER STUFF PROBABLY TO BE REMOVED}

Some other comments that we need to weave into a narrative eventually after later sections are written:
\begin{itemize}
\item
GPT versus GPT2.
What happens when we don't have enough data?
This is the main question, and we can use out metrics to evaluate that, but we also get very different results for GPT versus GPT2.
\item
The spectral norm is a regularizer, used to distinguish good-better-best, not a quality metric.
For example, it can ``collapse,'' and for bad models we can have small spectral norm.
So, it isn't really a quality metric.
\item
One question that isn't obvious is whether regularization metrics can be used as quality metrics.
One might think so, but the answer isn't obviously yes.
We show that the answer is No.
A regularizer is designed to select a unique solution from a non-unique good-better-best.
Quality metrics can also distinguish good versus bad.
\item
(We should at least mention this is like the statistical thing where we evaluate which model is better, as oposed to asking if a given model is good, I forget the name of that.)
\item
There are cases where the model is bad but regularization metric doesn't tell you that.
Quality should be correlated in an empirical way.
Correlated with good-better-best; but also tell good-bad.
\item
Question: why not use regularier for quality?
Answer: A regularizer selects from a given set of degenerate models one which is nice or unique.
It doesn't tell good versus bad, i.e., whether that model class is any good.
\item
Thus, it isn't obvious that norm-based metrics should do well, and they don't in general.
\item
We give examples of all of these: bad data; defective data; and distill models in a bad way.
(Of course, bad data means bad model, at least indirectly, since the quality of the data affects the properties of the model.)
\item
We can select a model and change it, i.e., we don't just do hyperparameter fiddling.
\end{itemize}



\end{document}
