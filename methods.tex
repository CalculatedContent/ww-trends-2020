section{Methods}
\label{sxn:methods}

We assume we are given several pretrained Deep Neural Networks (DNNs), as part of a similar architecture.
We would like to estimate the trends in the reported test / generalization accuracy accross a series of similar archtectures.  
For example, below we compare the 8 pretrained models in the VGG series: (VGG11, VGG11\_BN$\cdots$ VGG19), with
and without Batch Normalization, trained on ImageNet, and widely available in the pyTorch distribution.

To do this, we will compute a variety of \emph{Complexity Metrics} based on the Product Norm of the layer weight matrics.
Note that unlike traditional ML approaches, however, we do not seek a bound on the complexity (i.e. test error), 
nor are we trying to evaluating a single model with differing hyperparmeters.  We wish to examine different models a 
common architecture series. And, also, compare different architectures themselves.  

Let us write the Energy Landscape (or optimization function) for a typical DNN with $L$ layers as
\begin{equation}
E_{DNN}=h_{L}(\mathbf{W}_{L}\times h_{L-1}(\mathbf{W}_{L-1}\times h_{L-2}(\cdots)+\mathbf{b}_{L-1})+\mathbf{b}_{L})  .
\label{eqn:dnn_energy}
\end{equation}

with activation functions $h_{l}(\cdot)$,  weight matrices $\mathbf{W}_{l}$, and the biases $\mathbf{b}_{l}$.

The model has been (or will be) trained on (unspecified) labeled data $\{d_{i},y_{i}\}\in\mathcal{D}$, 
using Backprop, by minimizing some (also unspecified) loss function $\mathcal{L}()$.  Moreover, we expect that most well trained,. production quality models will employ 1 or more forms of on regularization, such as Batch Normalization, Dropout, etc, and will also contain additional structure such as Skip Connections etc. Here, we ignore these details, and focus only on the weight matrices. 

Each layer contains by one or more layer 2D weight matrices $\mathbf{W}_{L}$, and/or the 2D feature maps $\mathbf{W}_{i,L}$ extracted from 2D Convolutional layers.  (For notational convenience, we may drop the $i$ and/or $i,l$ subscripts below.) We assume the layer weight matrices are statistically independent, allowing us to estimate the Complexity $\mathcal{C}$, or test accuracy, with a standard Product Norm, which resembles a data dependent VC complexity

\begin{equation}
\mathcal{C}\sim\Vert\mathbf{W}_{1}\Vert\times\Vert\mathbf{W}_{2}\Vert\cdots\Vert\mathbf{W}_{L}\Vert ,
\end{equation}
where $\mathbf{W}$ is an $(N\times M)$ weight matrix, with $N\ge M$, and 
 $\Vert\mathbf{W}\Vert$ is a matrix norm.   We will actually compute the log Complexity, which takes the form 
of an Average Log Norm:
\begin{eqnarray*}
\log\mathcal{C} &\sim& \log\Vert\mathbf{W}_{1}\Vert+\log\Vert\mathbf{W}_{2}\Vert\cdots\log\Vert\mathbf{W}_{L}\Vert
\end{eqnarray*}

Here, we will consider the following Norms:

\begin{itemize}
 \item Frobenius Norm: $\Vert\mathbf{W}\Vert^{2}_{F}=\Vert\mathbf{W}\Vert^{2}_{2}=\sum_{i,j}w^{2}_{i,j}$
 \item Spectral Norm:  $\Vert\mathbf{W}\Vert_{\infty}=\lambda_{max}$
 \item $\alpha-$Norm (or Shatten Norm) $\Vert\mathbf{W}\Vert^{\alpha}_{\alpha}=\sum_{i=1}^{M}\lambda^{\alpha}$,
\end{itemize}

where $\lambda_{i}$ is the $i-th$ eigenvalue of the correlation matrix $\mathbf{X}=\dfrac{1}{N}\mathbf{W}^{T}\mathbf{W}$, and $\lambda_{max}$ is the maximum eigenvalue.   (Note that eigenvalues are the square of the singular values $\sigma_{i}$ of $\mathbf{W}$ :  $\lambda_{i}=\sigma^{2}_{i}$.)

The exponent $\alpha$ is the power law exponent that arises in our \emph{Theory of Heavy Tailed Self Regularization}, and is the determined by fitting the Empirical Spectral Density (ESD) of $\mathbf{X}$--i.e. a histogram of the eigenvalues--$\rho(\lambda)$ to a truncated power law

\begin{equation}
\rho(\lambda)\sim\lambda^{\alpha},\;\;\lambda\le\lambda_{max}
\end{equation}

We will also consider an approximate capacity metric, $\hat{\alpha}$, shown previously to correlate well with the trends in reported test accuracies of pretrained DNNs \cite{MM}

\begin{itemize}
 \item $\hat{\alpha}=\alpha\log\lambda_{max}\approx\log\Vert\mathbf{W}\Vert^{\alpha}_{\alpha}$
\end{itemize}

which approximates the log $\alpha-$Norm for both Very Heavy Tailed weight matrices ($alpha < 2$  and reasoably well for finite size, Moderately Heavy Tailed ones $\alpha\in[2,5]$.  


SOME MORE DISCUSSION HERE

\charles{not actually feature Maps, need to find the correct term: Tensor Slice}

\paragraph{Spectral Analysis of Convolutional 2D Layers}
While we can easily analyze Linear layers, there is some ambiguity in performing spectral analysis on Convolutional 2D (Conv2D) layers.  A Conv2D layer is a 4-index tensor of dimension $(w,h,in\_ch,out\_ch)$, specified by an $(w\times h)$ kernel, and $in\_ch,out\_ch$ input and output channels, respectively.  Typically, the $w=h=k$, giving $(k\times k)$ Tensor Slices $\mathbf{W}_{i,L}$
of dimension $(in\_ch\times out\_ch)$ each.  Usually $in\_ch\ll out\_ch$.
There are at least 3 different approaches to computing the Singular Values Decomposition(s) (SVD) of an Conv2D layer
\begin{enumerate}
\item run SVD on each of the Tensor Slices $\mathbf{W}_{i,L}$, yielding $(k\times k)$ sets of $M$ singular values. 
\item stack the feature maps into a single rectangular matrix of, say, dimension $((k\times k\times out\_ch)\times in\_ch)$, yielding $in\_ch$ singular values
\item Compute the 2D Fourier Transform (FFT) for each of the $(in\_ch, out\_ch)$ pairs, and run SVD on the resulting Fourier coeffients\cite{Long2019}.  This leads to $\sim(k\times in\_ch\times out\_ch)$ non-zero singular values.
\end{enumerate}

Each method has tradeoffs.  While, in principle, 3. is mathematically sound, the actual data available
in a pretrained model is usually the ourput of the Conv2D transformation, followed by Pooling and/or Batch Nprmalization,
and this complicates the interpretation.  For this study, because we are computing tens of thousands of calculations,  
we select $1.$, which is numerically the fastest and easiest to reproduce.\footnote{We will provide a Google Colab notebook where all results can be reproduced, with the option to redo the calculations with option 3 for the SVD of the Conv2D.}
----








